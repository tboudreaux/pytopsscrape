%% Generated by Sphinx.
\def\sphinxdocclass{report}
\documentclass[letterpaper,10pt,english]{sphinxmanual}
\ifdefined\pdfpxdimen
   \let\sphinxpxdimen\pdfpxdimen\else\newdimen\sphinxpxdimen
\fi \sphinxpxdimen=.75bp\relax
\ifdefined\pdfimageresolution
    \pdfimageresolution= \numexpr \dimexpr1in\relax/\sphinxpxdimen\relax
\fi
%% let collapsable pdf bookmarks panel have high depth per default
\PassOptionsToPackage{bookmarksdepth=5}{hyperref}

\PassOptionsToPackage{warn}{textcomp}
\usepackage[utf8]{inputenc}
\ifdefined\DeclareUnicodeCharacter
% support both utf8 and utf8x syntaxes
  \ifdefined\DeclareUnicodeCharacterAsOptional
    \def\sphinxDUC#1{\DeclareUnicodeCharacter{"#1}}
  \else
    \let\sphinxDUC\DeclareUnicodeCharacter
  \fi
  \sphinxDUC{00A0}{\nobreakspace}
  \sphinxDUC{2500}{\sphinxunichar{2500}}
  \sphinxDUC{2502}{\sphinxunichar{2502}}
  \sphinxDUC{2514}{\sphinxunichar{2514}}
  \sphinxDUC{251C}{\sphinxunichar{251C}}
  \sphinxDUC{2572}{\textbackslash}
\fi
\usepackage{cmap}
\usepackage[T1]{fontenc}
\usepackage{amsmath,amssymb,amstext}
\usepackage{babel}



\usepackage{tgtermes}
\usepackage{tgheros}
\renewcommand{\ttdefault}{txtt}



\usepackage[Bjarne]{fncychap}
\usepackage{sphinx}

\fvset{fontsize=auto}
\usepackage{geometry}


% Include hyperref last.
\usepackage{hyperref}
% Fix anchor placement for figures with captions.
\usepackage{hypcap}% it must be loaded after hyperref.
% Set up styles of URL: it should be placed after hyperref.
\urlstyle{same}


\usepackage{sphinxmessages}
\setcounter{tocdepth}{5}
\setcounter{secnumdepth}{5}


\title{pyTOPSScrape}
\date{Sep 06, 2022}
\release{0.5}
\author{Thomas Boudreaux}
\newcommand{\sphinxlogo}{\vbox{}}
\renewcommand{\releasename}{Release}
\makeindex
\begin{document}

\pagestyle{empty}
\sphinxmaketitle
\pagestyle{plain}
\sphinxtableofcontents
\pagestyle{normal}
\phantomsection\label{\detokenize{index::doc}}

\index{module@\spxentry{module}!pyTOPSScrape.scripts@\spxentry{pyTOPSScrape.scripts}}\index{pyTOPSScrape.scripts@\spxentry{pyTOPSScrape.scripts}!module@\spxentry{module}}

\chapter{pyTOPSScrape.api package}
\label{\detokenize{pyTOPSScrape.api:pytopsscrape-api-package}}\label{\detokenize{pyTOPSScrape.api::doc}}

\section{Submodules}
\label{\detokenize{pyTOPSScrape.api:submodules}}

\section{pyTOPSScrape.api.api module}
\label{\detokenize{pyTOPSScrape.api:module-pyTOPSScrape.api.api}}\label{\detokenize{pyTOPSScrape.api:pytopsscrape-api-api-module}}\index{module@\spxentry{module}!pyTOPSScrape.api.api@\spxentry{pyTOPSScrape.api.api}}\index{pyTOPSScrape.api.api@\spxentry{pyTOPSScrape.api.api}!module@\spxentry{module}}
\sphinxAtStartPar
\sphinxstylestrong{Author:} Thomas M. Boudreaux

\sphinxAtStartPar
\sphinxstylestrong{Created:} September 2021

\sphinxAtStartPar
\sphinxstylestrong{Last Modified:} September 2021

\sphinxAtStartPar
Psuedo API for querying TOPS webform
\index{TOPS\_query() (in module pyTOPSScrape.api.api)@\spxentry{TOPS\_query()}\spxextra{in module pyTOPSScrape.api.api}}

\begin{fulllineitems}
\phantomsection\label{\detokenize{pyTOPSScrape.api:pyTOPSScrape.api.api.TOPS_query}}\pysiglinewithargsret{\sphinxcode{\sphinxupquote{pyTOPSScrape.api.api.}}\sphinxbfcode{\sphinxupquote{TOPS\_query}}}{\emph{\DUrole{n}{mixString}\DUrole{p}{:} \DUrole{n}{str}}, \emph{\DUrole{n}{mixName}\DUrole{p}{:} \DUrole{n}{str}}, \emph{\DUrole{n}{nAttempts}\DUrole{p}{:} \DUrole{n}{int}}}{{ $\rightarrow$ bytes}}
\sphinxAtStartPar
Query TOPS form and retry n times
\begin{quote}\begin{description}
\item[{Parameters}] \leavevmode\begin{itemize}
\item {} 
\sphinxAtStartPar
\sphinxstyleliteralstrong{\sphinxupquote{mixString}} (\sphinxstyleliteralemphasis{\sphinxupquote{string}}) \textendash{} string in the form of: “massFrac0 Element0 massFrac1 Element1 …”
which will be sumbittedi in the webform for mixture

\item {} 
\sphinxAtStartPar
\sphinxstyleliteralstrong{\sphinxupquote{mixName}} (\sphinxstyleliteralemphasis{\sphinxupquote{string}}) \textendash{} name to be used in the webform

\item {} 
\sphinxAtStartPar
\sphinxstyleliteralstrong{\sphinxupquote{nAttemptes}} (\sphinxstyleliteralemphasis{\sphinxupquote{int}}) \textendash{} How many times to reattempt after a failure.

\end{itemize}

\item[{Returns}] \leavevmode
\sphinxAtStartPar
\sphinxstylestrong{tableHTML} \textendash{} Table quired from TOPS cite.

\item[{Return type}] \leavevmode
\sphinxAtStartPar
bytes

\end{description}\end{quote}

\end{fulllineitems}

\index{TOPS\_query\_async\_distributor() (in module pyTOPSScrape.api.api)@\spxentry{TOPS\_query\_async\_distributor()}\spxextra{in module pyTOPSScrape.api.api}}

\begin{fulllineitems}
\phantomsection\label{\detokenize{pyTOPSScrape.api:pyTOPSScrape.api.api.TOPS_query_async_distributor}}\pysiglinewithargsret{\sphinxcode{\sphinxupquote{pyTOPSScrape.api.api.}}\sphinxbfcode{\sphinxupquote{TOPS\_query\_async\_distributor}}}{\emph{\DUrole{n}{compList}}, \emph{\DUrole{n}{outputDirectory}}, \emph{\DUrole{n}{njobs}\DUrole{o}{=}\DUrole{default_value}{10}}}{}
\sphinxAtStartPar
Distributes TOPS query jobs to different threads and gathers the results
together. Writes out output.
\begin{quote}\begin{description}
\item[{Parameters}] \leavevmode\begin{itemize}
\item {} 
\sphinxAtStartPar
\sphinxstyleliteralstrong{\sphinxupquote{aFiles}} (\sphinxstyleliteralemphasis{\sphinxupquote{list of TextIO}}) \textendash{} List of file like objects to abundance files to be parsed

\item {} 
\sphinxAtStartPar
\sphinxstyleliteralstrong{\sphinxupquote{outputDirectory}} (\sphinxstyleliteralemphasis{\sphinxupquote{str}}) \textendash{} Path to directory to save TOPS query results to.

\item {} 
\sphinxAtStartPar
\sphinxstyleliteralstrong{\sphinxupquote{njobs}} (\sphinxstyleliteralemphasis{\sphinxupquote{int}}\sphinxstyleliteralemphasis{\sphinxupquote{, }}\sphinxstyleliteralemphasis{\sphinxupquote{default=10}}) \textendash{} Number of concurrent jobs to allow at a time.

\end{itemize}

\end{description}\end{quote}

\end{fulllineitems}

\index{call() (in module pyTOPSScrape.api.api)@\spxentry{call()}\spxextra{in module pyTOPSScrape.api.api}}

\begin{fulllineitems}
\phantomsection\label{\detokenize{pyTOPSScrape.api:pyTOPSScrape.api.api.call}}\pysiglinewithargsret{\sphinxcode{\sphinxupquote{pyTOPSScrape.api.api.}}\sphinxbfcode{\sphinxupquote{call}}}{\emph{\DUrole{n}{aMap}\DUrole{p}{:} \DUrole{n}{str}}, \emph{\DUrole{n}{aTable}\DUrole{p}{:} \DUrole{n}{str}}, \emph{\DUrole{n}{outputDir}\DUrole{p}{:} \DUrole{n}{str}}, \emph{\DUrole{n}{jobs}\DUrole{p}{:} \DUrole{n}{int}}}{}
\sphinxAtStartPar
Main TOPS psuedo API call function. Will save results to outputDir with
file format OP:IDX\_X\_Y\_Z.dat where IDX is the ID of the composition (parallel
to DSEP composition ID), X is the classical Hydrogen mass fraction, Y  is
the classical Helium mass fraction, and Z is the classical metal mass
fraction.
\begin{quote}\begin{description}
\item[{Parameters}] \leavevmode\begin{itemize}
\item {} 
\sphinxAtStartPar
\sphinxstyleliteralstrong{\sphinxupquote{aMap}} (\sphinxstyleliteralemphasis{\sphinxupquote{str}}) \textendash{} Path to the list of classical compositions to be used. List should
be given as an ascii file where ecach row is X,Y,Z

\item {} 
\sphinxAtStartPar
\sphinxstyleliteralstrong{\sphinxupquote{aTable}} (\sphinxstyleliteralemphasis{\sphinxupquote{str}}) \textendash{} Path to chemical abundance table to be used as base composition.

\item {} 
\sphinxAtStartPar
\sphinxstyleliteralstrong{\sphinxupquote{outputDir}} (\sphinxstyleliteralemphasis{\sphinxupquote{str}}) \textendash{} Path to directory save TOPS query results into

\item {} 
\sphinxAtStartPar
\sphinxstyleliteralstrong{\sphinxupquote{jobs}} (\sphinxstyleliteralemphasis{\sphinxupquote{int}}) \textendash{} Number of threads to query TOPS webform on

\end{itemize}

\end{description}\end{quote}

\end{fulllineitems}

\index{parse\_table() (in module pyTOPSScrape.api.api)@\spxentry{parse\_table()}\spxextra{in module pyTOPSScrape.api.api}}

\begin{fulllineitems}
\phantomsection\label{\detokenize{pyTOPSScrape.api:pyTOPSScrape.api.api.parse_table}}\pysiglinewithargsret{\sphinxcode{\sphinxupquote{pyTOPSScrape.api.api.}}\sphinxbfcode{\sphinxupquote{parse\_table}}}{\emph{\DUrole{n}{html}\DUrole{p}{:} \DUrole{n}{bytes}}}{{ $\rightarrow$ str}}
\sphinxAtStartPar
Parse the bytes table returned from mechanize into a string
\begin{quote}\begin{description}
\item[{Parameters}] \leavevmode
\sphinxAtStartPar
\sphinxstyleliteralstrong{\sphinxupquote{html}} (\sphinxstyleliteralemphasis{\sphinxupquote{bytes}}) \textendash{} bytes table retuend from mechanize bowser at second TOPS submission
form

\item[{Returns}] \leavevmode
\sphinxAtStartPar
\sphinxstylestrong{table} \textendash{} parsed html soruce in the form of a string

\item[{Return type}] \leavevmode
\sphinxAtStartPar
string

\end{description}\end{quote}

\end{fulllineitems}

\index{query\_and\_parse() (in module pyTOPSScrape.api.api)@\spxentry{query\_and\_parse()}\spxextra{in module pyTOPSScrape.api.api}}

\begin{fulllineitems}
\phantomsection\label{\detokenize{pyTOPSScrape.api:pyTOPSScrape.api.api.query_and_parse}}\pysiglinewithargsret{\sphinxcode{\sphinxupquote{pyTOPSScrape.api.api.}}\sphinxbfcode{\sphinxupquote{query\_and\_parse}}}{\emph{\DUrole{n}{compList}\DUrole{p}{:} \DUrole{n}{list}}, \emph{\DUrole{n}{outputDirectory}\DUrole{p}{:} \DUrole{n}{int}}, \emph{\DUrole{n}{i}\DUrole{p}{:} \DUrole{n}{int}}, \emph{\DUrole{n}{nAttempts}\DUrole{p}{:} \DUrole{n}{int} \DUrole{o}{=} \DUrole{default_value}{10}}}{}
\sphinxAtStartPar
Async coroutine to query TOPS webform, parse the output, and write that
to disk.
\begin{quote}\begin{description}
\item[{Parameters}] \leavevmode\begin{itemize}
\item {} 
\sphinxAtStartPar
\sphinxstyleliteralstrong{\sphinxupquote{file}} (\sphinxstyleliteralemphasis{\sphinxupquote{TextIO}}) \textendash{} Abundance file to be parsed for the form as defined in the docstring
for the function parse\_numfrac\_file

\item {} 
\sphinxAtStartPar
\sphinxstyleliteralstrong{\sphinxupquote{outputDirectory}} (\sphinxstyleliteralemphasis{\sphinxupquote{str}}) \textendash{} Path to write out results of TOPS webquery

\item {} 
\sphinxAtStartPar
\sphinxstyleliteralstrong{\sphinxupquote{i}} (\sphinxstyleliteralemphasis{\sphinxupquote{int}}) \textendash{} Index of composition so file name can properly keep track of where
it is, even in parallel processing.

\item {} 
\sphinxAtStartPar
\sphinxstyleliteralstrong{\sphinxupquote{nAttempts}} (\sphinxstyleliteralemphasis{\sphinxupquote{int}}\sphinxstyleliteralemphasis{\sphinxupquote{, }}\sphinxstyleliteralemphasis{\sphinxupquote{default=10}}) \textendash{} Number of time to retry TOPS query before failing out

\end{itemize}

\end{description}\end{quote}

\end{fulllineitems}

\index{submit\_TOPS\_form() (in module pyTOPSScrape.api.api)@\spxentry{submit\_TOPS\_form()}\spxextra{in module pyTOPSScrape.api.api}}

\begin{fulllineitems}
\phantomsection\label{\detokenize{pyTOPSScrape.api:pyTOPSScrape.api.api.submit_TOPS_form}}\pysiglinewithargsret{\sphinxcode{\sphinxupquote{pyTOPSScrape.api.api.}}\sphinxbfcode{\sphinxupquote{submit\_TOPS\_form}}}{\emph{\DUrole{n}{mixString}\DUrole{p}{:} \DUrole{n}{str}}, \emph{\DUrole{n}{mixName}\DUrole{p}{:} \DUrole{n}{str}}, \emph{\DUrole{n}{massFrac}\DUrole{p}{:} \DUrole{n}{bool} \DUrole{o}{=} \DUrole{default_value}{True}}}{{ $\rightarrow$ bytes}}
\sphinxAtStartPar
Open the Los Alamos opacity website, submit a given composition and then
return the resultant table.
\begin{quote}\begin{description}
\item[{Parameters}] \leavevmode\begin{itemize}
\item {} 
\sphinxAtStartPar
\sphinxstyleliteralstrong{\sphinxupquote{mixString}} (\sphinxstyleliteralemphasis{\sphinxupquote{string}}) \textendash{} string in the form of: “massFrac0 Element0 massFrac1 Element1 …”
which will be sumbittedi in the webform for mixture

\item {} 
\sphinxAtStartPar
\sphinxstyleliteralstrong{\sphinxupquote{mixName}} (\sphinxstyleliteralemphasis{\sphinxupquote{string}}) \textendash{} name to be used in the webform

\item {} 
\sphinxAtStartPar
\sphinxstyleliteralstrong{\sphinxupquote{massFrac}} (\sphinxstyleliteralemphasis{\sphinxupquote{bool}}\sphinxstyleliteralemphasis{\sphinxupquote{, }}\sphinxstyleliteralemphasis{\sphinxupquote{default=True}}) \textendash{} Submit as massFrac instead of numberFrac

\end{itemize}

\item[{Returns}] \leavevmode
\sphinxAtStartPar
\sphinxstylestrong{tableHTML} \textendash{} Table quired from TOPS cite.

\item[{Return type}] \leavevmode
\sphinxAtStartPar
bytes

\end{description}\end{quote}

\end{fulllineitems}



\section{pyTOPSScrape.api.convert module}
\label{\detokenize{pyTOPSScrape.api:module-pyTOPSScrape.api.convert}}\label{\detokenize{pyTOPSScrape.api:pytopsscrape-api-convert-module}}\index{module@\spxentry{module}!pyTOPSScrape.api.convert@\spxentry{pyTOPSScrape.api.convert}}\index{pyTOPSScrape.api.convert@\spxentry{pyTOPSScrape.api.convert}!module@\spxentry{module}}
\sphinxAtStartPar
\sphinxstylestrong{Author:} Thomas M. Boudreaux

\sphinxAtStartPar
\sphinxstylestrong{Created:} September 2021

\sphinxAtStartPar
\sphinxstylestrong{Last Modified:} September 2021

\sphinxAtStartPar
Main conversion code for TOPS api, responsible for takine many TOPS results and
merging them into a single OPAL formate high temperature opacity file.


\subsection{Functions}
\label{\detokenize{pyTOPSScrape.api:functions}}\begin{itemize}
\item {} \begin{description}
\item[{comp\_list\_2\_dict}] \leavevmode
\sphinxAtStartPar
Take a list containing compsoition information for a star in the form of
{[}(‘Element Symbol’, massFraction, numberFraction),…{]} and convert that
into a dictionary of the form:
\{‘Element Symbol’: (massFraction, numberFraction),..\}.

\end{description}

\item {} \begin{description}
\item[{parse\_RMO\_TOPS\_table\_file}] \leavevmode
\sphinxAtStartPar
Given the path to a file queried from the TOPS webform put it into a
computer usable form of 3 arrays. One array of mass density, one of
LogT and one of log Rossland Mean Opacity

\end{description}

\item {} \begin{description}
\item[{convert\_rho\_2\_LogR}] \leavevmode
\sphinxAtStartPar
Maps a given kappa(rho,logT) parameter space onto a kappa(LogR, LogT) field
through interpolation. The final field is the field that DSEP needs.

\end{description}

\item {} \begin{description}
\item[{extract\_composition\_path}] \leavevmode
\sphinxAtStartPar
Given the name of a TOPS return file (named in the format OP:n\_X\_Y\_Z.dat)
extract X, Y, and Z

\end{description}

\item {} \begin{description}
\item[{format\_opal\_comp\_table}] \leavevmode
\sphinxAtStartPar
Take in all the information from a given TOPS tables and format it
to the proper format for DSEP to undersand. Leave in some placeholders
so that in future table can be labeld as the proper number.

\end{description}

\item {} \begin{description}
\item[{format\_OPAL\_header}] \leavevmode
\sphinxAtStartPar
Writes the header of the opacity table that DSEP expects. This is written
to be the same length (and basically the same contents) of the header
from the OPACITY project. Not sure if that is required; however, if so
I am matching it.

\end{description}

\item {} \begin{description}
\item[{format\_OPAL\_table}] \leavevmode
\sphinxAtStartPar
Given a dictionary of tables and a composition Dictionary for solar
composition in a given mixture (AGSS08, GS98, etc…) merge all the
information together into a string which can be written to disk and
would be the format of an opacoty project table (what DSEP expects)

\end{description}

\item {} \begin{description}
\item[{format\_TOPS\_to\_OPAL}] \leavevmode
\sphinxAtStartPar
Take the path to a table queried from the TOPS web form and fully convert it
into a table which can be directly read by DSEP. (Note this function
does not write anything to disk; however, the return products can be
written to disk)

\end{description}

\item {} \begin{description}
\item[{rebuild\_formated\_table}] \leavevmode
\sphinxAtStartPar
Iterate over a list of opacity tables and a list of desired chemical
compositions then replace the contents of the table list with the newly
updated RMOs frmo the interpolation.

\end{description}

\item {} \begin{description}
\item[{TOPS\_2\_OPAL}] \leavevmode
\sphinxAtStartPar
Main conversoin utility to go between some set of TOPS tables and an OPAl
table. Will take a set of 126 TOPS tables where each one is the opacity for
one composition over a number of temperature and densities and rearange them
into one large file with 126 tables within it. Each table will be over a
range of temperatures and R values. To get to R val interpolation is used.

\end{description}

\end{itemize}
\index{TOPS\_2\_OPAL() (in module pyTOPSScrape.api.convert)@\spxentry{TOPS\_2\_OPAL()}\spxextra{in module pyTOPSScrape.api.convert}}

\begin{fulllineitems}
\phantomsection\label{\detokenize{pyTOPSScrape.api:pyTOPSScrape.api.convert.TOPS_2_OPAL}}\pysiglinewithargsret{\sphinxcode{\sphinxupquote{pyTOPSScrape.api.convert.}}\sphinxbfcode{\sphinxupquote{TOPS\_2\_OPAL}}}{\emph{\DUrole{n}{outputDirectory}\DUrole{p}{:} \DUrole{n}{str}}, \emph{\DUrole{n}{aTable}\DUrole{p}{:} \DUrole{n}{str}}, \emph{\DUrole{n}{aMap}\DUrole{p}{:} \DUrole{n}{str}}, \emph{\DUrole{n}{output}\DUrole{p}{:} \DUrole{n}{str}}, \emph{\DUrole{n}{nonRect}\DUrole{p}{:} \DUrole{n}{bool} \DUrole{o}{=} \DUrole{default_value}{False}}}{}
\sphinxAtStartPar
Main conversoin utility to go between some set of TOPS tables and an OPAl
table. Will take a set of 126 TOPS tables where each one is the opacity for
one composition over a number of temperature and densities and rearange them
into one large file with 126 tables within it. Each table will be over a
range of temperatures and R values. To get to R val interpolation is used.
\begin{quote}\begin{description}
\item[{Parameters}] \leavevmode\begin{itemize}
\item {} 
\sphinxAtStartPar
\sphinxstyleliteralstrong{\sphinxupquote{outputDirectory}} (\sphinxstyleliteralemphasis{\sphinxupquote{str}}) \textendash{} Path to directory where TOPS query results are stored

\item {} 
\sphinxAtStartPar
\sphinxstyleliteralstrong{\sphinxupquote{aTable}} (\sphinxstyleliteralemphasis{\sphinxupquote{str}}) \textendash{} Path to a reference abundance table to use when filling the header
with compositional information

\item {} 
\sphinxAtStartPar
\sphinxstyleliteralstrong{\sphinxupquote{aMap}} (\sphinxstyleliteralemphasis{\sphinxupquote{str}}) \textendash{} Path to the abundance map. This should be an ascii file where each
row is X,Y,Z. Each row will correspond to one rescaled composition
which will be queried.

\item {} 
\sphinxAtStartPar
\sphinxstyleliteralstrong{\sphinxupquote{output}} (\sphinxstyleliteralemphasis{\sphinxupquote{str}}) \textendash{} Path to save final OPAL formated table too

\item {} 
\sphinxAtStartPar
\sphinxstyleliteralstrong{\sphinxupquote{nonRect}} (\sphinxstyleliteralemphasis{\sphinxupquote{bool}}\sphinxstyleliteralemphasis{\sphinxupquote{, }}\sphinxstyleliteralemphasis{\sphinxupquote{default=False}}) \textendash{} Flag to control whether output tables will be rectangular or have
their corners cut off in a way consistant which how DSEP expects
OPAL tables.

\end{itemize}

\end{description}\end{quote}

\end{fulllineitems}

\index{comp\_list\_2\_dict() (in module pyTOPSScrape.api.convert)@\spxentry{comp\_list\_2\_dict()}\spxextra{in module pyTOPSScrape.api.convert}}

\begin{fulllineitems}
\phantomsection\label{\detokenize{pyTOPSScrape.api:pyTOPSScrape.api.convert.comp_list_2_dict}}\pysiglinewithargsret{\sphinxcode{\sphinxupquote{pyTOPSScrape.api.convert.}}\sphinxbfcode{\sphinxupquote{comp\_list\_2\_dict}}}{\emph{\DUrole{n}{compList}\DUrole{p}{:} \DUrole{n}{collections.abc.Iterator}}}{{ $\rightarrow$ dict}}
\sphinxAtStartPar
Take a list containing compsoition information for a star in the form of
{[}(‘Element Symbol’, massFraction, numberFraction),…{]} and convert that
into a dictionary of the form:
\{‘Element Symbol’: (massFraction, numberFraction),..\}.
\begin{quote}\begin{description}
\item[{Parameters}] \leavevmode
\sphinxAtStartPar
\sphinxstyleliteralstrong{\sphinxupquote{compList}} (\sphinxstyleliteralemphasis{\sphinxupquote{Iterator}}) \textendash{} list of the form: {[}(ElementSymbol, massFrac, numberFrac, ZmassFrac,
massFrac Uncertanity, numberFrac Uncertanity, ZMassFrac
Uncertanity),…{]}

\item[{Returns}] \leavevmode
\sphinxAtStartPar
Dictionary of the form \{‘Element’:(massFrac, numberFrac, ZmassFrac,
massFrac Uncertanity, numberFrac Uncertanity, ZMassFrac
Uncertanity),…\}

\item[{Return type}] \leavevmode
\sphinxAtStartPar
dict

\end{description}\end{quote}

\end{fulllineitems}

\index{convert\_rho\_2\_LogR() (in module pyTOPSScrape.api.convert)@\spxentry{convert\_rho\_2\_LogR()}\spxextra{in module pyTOPSScrape.api.convert}}

\begin{fulllineitems}
\phantomsection\label{\detokenize{pyTOPSScrape.api:pyTOPSScrape.api.convert.convert_rho_2_LogR}}\pysiglinewithargsret{\sphinxcode{\sphinxupquote{pyTOPSScrape.api.convert.}}\sphinxbfcode{\sphinxupquote{convert\_rho\_2\_LogR}}}{\emph{\DUrole{n}{rho}\DUrole{p}{:} \DUrole{n}{numpy.ndarray}}, \emph{\DUrole{n}{LogT}\DUrole{p}{:} \DUrole{n}{numpy.ndarray}}, \emph{\DUrole{n}{RMO}\DUrole{p}{:} \DUrole{n}{numpy.ndarray}}}{{ $\rightarrow$ Tuple\DUrole{p}{{[}}numpy.ndarray\DUrole{p}{, }numpy.ndarray\DUrole{p}{, }numpy.ndarray\DUrole{p}{{]}}}}
\sphinxAtStartPar
Maps a given kappa(rho,logT) parameter space onto a kappa(LogR, LogT) field
through interpolation. The final field is the field that DSEP needs.
\begin{quote}\begin{description}
\item[{Parameters}] \leavevmode\begin{itemize}
\item {} 
\sphinxAtStartPar
\sphinxstyleliteralstrong{\sphinxupquote{rho}} (\sphinxstyleliteralemphasis{\sphinxupquote{np.ndarray}}) \textendash{} mass density array of size n

\item {} 
\sphinxAtStartPar
\sphinxstyleliteralstrong{\sphinxupquote{LogT}} (\sphinxstyleliteralemphasis{\sphinxupquote{np.ndarray}}) \textendash{} LogT array of size m

\item {} 
\sphinxAtStartPar
\sphinxstyleliteralstrong{\sphinxupquote{RMO}} (\sphinxstyleliteralemphasis{\sphinxupquote{np.ndarray}}) \textendash{} Opacity Array of size m x n

\end{itemize}

\item[{Returns}] \leavevmode
\sphinxAtStartPar
\begin{itemize}
\item {} 
\sphinxAtStartPar
\sphinxstylestrong{targetLogR} (\sphinxstyleemphasis{np.ndarray(shape=19)}) \textendash{} Log R values which dsep requires

\item {} 
\sphinxAtStartPar
\sphinxstylestrong{targetLogR} (\sphinxstyleemphasis{np.ndarray(shape=70)}) \textendash{} Lof T values which dsep requires

\item {} 
\sphinxAtStartPar
\sphinxstylestrong{Opacity} (\sphinxstyleemphasis{np.ndarray(shape=(70, 19))}) \textendash{} Opacity array now interpolated into LogR, LogT space from rho LogT
space and sampled at the exact LogR and LogT values required.

\end{itemize}


\end{description}\end{quote}

\end{fulllineitems}

\index{extract\_composition\_path() (in module pyTOPSScrape.api.convert)@\spxentry{extract\_composition\_path()}\spxextra{in module pyTOPSScrape.api.convert}}

\begin{fulllineitems}
\phantomsection\label{\detokenize{pyTOPSScrape.api:pyTOPSScrape.api.convert.extract_composition_path}}\pysiglinewithargsret{\sphinxcode{\sphinxupquote{pyTOPSScrape.api.convert.}}\sphinxbfcode{\sphinxupquote{extract\_composition\_path}}}{\emph{\DUrole{n}{path}\DUrole{p}{:} \DUrole{n}{str}}}{{ $\rightarrow$ Tuple\DUrole{p}{{[}}float\DUrole{p}{, }float\DUrole{p}{, }float\DUrole{p}{{]}}}}
\sphinxAtStartPar
Given the name of a TOPS return file (named in the format OP:n\_X\_Y\_Z.dat)
extract X, Y, and Z
\begin{quote}\begin{description}
\item[{Parameters}] \leavevmode
\sphinxAtStartPar
\sphinxstyleliteralstrong{\sphinxupquote{path}} (\sphinxstyleliteralemphasis{\sphinxupquote{string}}) \textendash{} path to TOPS return file

\item[{Returns}] \leavevmode
\sphinxAtStartPar
\begin{itemize}
\item {} 
\sphinxAtStartPar
\sphinxstylestrong{X} (\sphinxstyleemphasis{float}) \textendash{} Hydrogen mass fraction

\item {} 
\sphinxAtStartPar
\sphinxstylestrong{Y} (\sphinxstyleemphasis{float}) \textendash{} Helium mass fraction

\item {} 
\sphinxAtStartPar
\sphinxstylestrong{Z} (\sphinxstyleemphasis{Metal mass fraction})

\end{itemize}


\end{description}\end{quote}

\end{fulllineitems}

\index{format\_OPAL\_header() (in module pyTOPSScrape.api.convert)@\spxentry{format\_OPAL\_header()}\spxextra{in module pyTOPSScrape.api.convert}}

\begin{fulllineitems}
\phantomsection\label{\detokenize{pyTOPSScrape.api:pyTOPSScrape.api.convert.format_OPAL_header}}\pysiglinewithargsret{\sphinxcode{\sphinxupquote{pyTOPSScrape.api.convert.}}\sphinxbfcode{\sphinxupquote{format\_OPAL\_header}}}{\emph{\DUrole{n}{compDict}\DUrole{p}{:} \DUrole{n}{dict}}}{{ $\rightarrow$ str}}
\sphinxAtStartPar
Writes the header of the opacity table that DSEP expects. This is written
to be the same length (and basically the same contents) of the header
from the OPACITY project. Not sure if that is required; however, if so
I am matching it.
\begin{quote}\begin{description}
\item[{Parameters}] \leavevmode
\sphinxAtStartPar
\sphinxstyleliteralstrong{\sphinxupquote{compDict}} (\sphinxstyleliteralemphasis{\sphinxupquote{dict}}) \textendash{} dictionary in the form: \{‘Element’: (massFrac, numFrac), …\} used
to fill up the header with composition information. This is meant
to be the “solar” compositon of whatever mix you are using so…
{[}Fe/H{]} = 0.0, {[}alpha/H{]} = 0.0, a(He) = 10.93

\item[{Returns}] \leavevmode
\sphinxAtStartPar
The Header to be prepended to the opacity table file

\item[{Return type}] \leavevmode
\sphinxAtStartPar
string

\end{description}\end{quote}

\end{fulllineitems}

\index{format\_OPAL\_table() (in module pyTOPSScrape.api.convert)@\spxentry{format\_OPAL\_table()}\spxextra{in module pyTOPSScrape.api.convert}}

\begin{fulllineitems}
\phantomsection\label{\detokenize{pyTOPSScrape.api:pyTOPSScrape.api.convert.format_OPAL_table}}\pysiglinewithargsret{\sphinxcode{\sphinxupquote{pyTOPSScrape.api.convert.}}\sphinxbfcode{\sphinxupquote{format\_OPAL\_table}}}{\emph{\DUrole{n}{tableDict}\DUrole{p}{:} \DUrole{n}{dict}}, \emph{\DUrole{n}{compDict}\DUrole{p}{:} \DUrole{n}{dict}}}{{ $\rightarrow$ str}}
\sphinxAtStartPar
Given a dictionary of tables and a composition Dictionary for solar
composition in a given mixture (AGSS08, GS98, etc…) merge all the
information together into a string which can be written to disk and
would be the format of an opacoty project table (what DSEP expects)
\begin{quote}\begin{description}
\item[{Parameters}] \leavevmode\begin{itemize}
\item {} 
\sphinxAtStartPar
\sphinxstyleliteralstrong{\sphinxupquote{tableDict}} (\sphinxstyleliteralemphasis{\sphinxupquote{dict}}) \textendash{} dictionary of table elements, containing a “Summary” entry
(metadata) and a “Table” entry. All occcurences of the the string
“TNUM” will be replaced with the index+1 of where that table occurs
in the file.

\item {} 
\sphinxAtStartPar
\sphinxstyleliteralstrong{\sphinxupquote{compDict}} (\sphinxstyleliteralemphasis{\sphinxupquote{dict}}) \textendash{} dictionary in the form: \{‘Element’: (massFrac, numFrac), …\} used
to fill up the header with composition information. This is meant
to be the “solar” compositon of whatever mix you are using so…
{[}Fe/H{]} = 0.0, {[}alpha/H{]} = 0.0, a(He) = 10.93

\end{itemize}

\item[{Returns}] \leavevmode
\sphinxAtStartPar
\sphinxstylestrong{OPALFormatted} \textendash{} Opacity Project formated table as a string which can be written to
disk

\item[{Return type}] \leavevmode
\sphinxAtStartPar
string

\end{description}\end{quote}

\end{fulllineitems}

\index{format\_TOPS\_to\_OPAL() (in module pyTOPSScrape.api.convert)@\spxentry{format\_TOPS\_to\_OPAL()}\spxextra{in module pyTOPSScrape.api.convert}}

\begin{fulllineitems}
\phantomsection\label{\detokenize{pyTOPSScrape.api:pyTOPSScrape.api.convert.format_TOPS_to_OPAL}}\pysiglinewithargsret{\sphinxcode{\sphinxupquote{pyTOPSScrape.api.convert.}}\sphinxbfcode{\sphinxupquote{format\_TOPS\_to\_OPAL}}}{\emph{\DUrole{n}{TOPSTable}\DUrole{p}{:} \DUrole{n}{str}}, \emph{\DUrole{n}{comp}\DUrole{p}{:} \DUrole{n}{tuple}}, \emph{\DUrole{n}{tnum}\DUrole{p}{:} \DUrole{n}{int}}, \emph{\DUrole{n}{upperNonRect}\DUrole{p}{:} \DUrole{n}{Optional\DUrole{p}{{[}}numpy.ndarray\DUrole{p}{{]}}} \DUrole{o}{=} \DUrole{default_value}{None}}}{{ $\rightarrow$ Tuple\DUrole{p}{{[}}str\DUrole{p}{, }float\DUrole{p}{, }float\DUrole{p}{, }float\DUrole{p}{, }numpy.ndarray\DUrole{p}{, }numpy.ndarray\DUrole{p}{, }numpy.ndarray\DUrole{p}{{]}}}}
\sphinxAtStartPar
Take the path to a table queried from the TOPS web form and fully convert it
into a table which can be directly read by DSEP. (Note this function
does not write anything to disk; however, the return products can be
written to disk)
\begin{quote}\begin{description}
\item[{Parameters}] \leavevmode\begin{itemize}
\item {} 
\sphinxAtStartPar
\sphinxstyleliteralstrong{\sphinxupquote{TOPSTable}} (\sphinxstyleliteralemphasis{\sphinxupquote{string}}) \textendash{} path to table queried from TOPS web form

\item {} 
\sphinxAtStartPar
\sphinxstyleliteralstrong{\sphinxupquote{comp}} (\sphinxstyleliteralemphasis{\sphinxupquote{dict}}) \textendash{} composition dictionary

\item {} 
\sphinxAtStartPar
\sphinxstyleliteralstrong{\sphinxupquote{tnum}} (\sphinxstyleliteralemphasis{\sphinxupquote{int}}) \textendash{} table number

\item {} 
\sphinxAtStartPar
\sphinxstyleliteralstrong{\sphinxupquote{upperNonRect}} (\sphinxstyleliteralemphasis{\sphinxupquote{np.ndarray}}\sphinxstyleliteralemphasis{\sphinxupquote{, }}\sphinxstyleliteralemphasis{\sphinxupquote{optional}}) \textendash{} array describing how to fill the top of the table non rectangurally

\end{itemize}

\item[{Returns}] \leavevmode
\sphinxAtStartPar
\begin{itemize}
\item {} 
\sphinxAtStartPar
\sphinxstylestrong{metaLine} (\sphinxstyleemphasis{str}) \textendash{} metadata extracted from table

\item {} 
\sphinxAtStartPar
\sphinxstylestrong{X} (\sphinxstyleemphasis{float}) \textendash{} Hydrogen mass fraction

\item {} 
\sphinxAtStartPar
\sphinxstylestrong{Y} (\sphinxstyleemphasis{float}) \textendash{} Helium mass fraction

\item {} 
\sphinxAtStartPar
\sphinxstylestrong{Z} (\sphinxstyleemphasis{float}) \textendash{} Metal mass fraction

\item {} 
\sphinxAtStartPar
\sphinxstylestrong{LogR} (\sphinxstyleemphasis{np.ndarray(shape=19)}) \textendash{} Log R values which dsep expects

\item {} 
\sphinxAtStartPar
\sphinxstylestrong{LogT} (\sphinxstyleemphasis{np.ndarray(shape=70)}) \textendash{} Log Temperature values which dsep expects.

\item {} 
\sphinxAtStartPar
\sphinxstylestrong{LogRMO} (\sphinxstyleemphasis{np.ndarray(shape=(70, 19))}) \textendash{} Log rossland mean opacities for the LogT and LogR arrays

\end{itemize}


\end{description}\end{quote}

\end{fulllineitems}

\index{format\_opal\_comp\_table() (in module pyTOPSScrape.api.convert)@\spxentry{format\_opal\_comp\_table()}\spxextra{in module pyTOPSScrape.api.convert}}

\begin{fulllineitems}
\phantomsection\label{\detokenize{pyTOPSScrape.api:pyTOPSScrape.api.convert.format_opal_comp_table}}\pysiglinewithargsret{\sphinxcode{\sphinxupquote{pyTOPSScrape.api.convert.}}\sphinxbfcode{\sphinxupquote{format\_opal\_comp\_table}}}{\emph{\DUrole{n}{LogR}\DUrole{p}{:} \DUrole{n}{numpy.ndarray}}, \emph{\DUrole{n}{LogT}\DUrole{p}{:} \DUrole{n}{numpy.ndarray}}, \emph{\DUrole{n}{LogRMO}\DUrole{p}{:} \DUrole{n}{numpy.ndarray}}, \emph{\DUrole{n}{TNUM}\DUrole{p}{:} \DUrole{n}{int}}, \emph{\DUrole{n}{comp}\DUrole{p}{:} \DUrole{n}{Optional\DUrole{p}{{[}}dict\DUrole{p}{{]}}} \DUrole{o}{=} \DUrole{default_value}{None}}, \emph{\DUrole{n}{upperNonRect}\DUrole{p}{:} \DUrole{n}{Optional\DUrole{p}{{[}}numpy.ndarray\DUrole{p}{{]}}} \DUrole{o}{=} \DUrole{default_value}{None}}, \emph{\DUrole{n}{nonRect}\DUrole{p}{:} \DUrole{n}{bool} \DUrole{o}{=} \DUrole{default_value}{False}}}{{ $\rightarrow$ Tuple\DUrole{p}{{[}}str\DUrole{p}{, }str\DUrole{p}{{]}}}}
\sphinxAtStartPar
Take in all the information from a given TOPS tables and format it
to the proper format for DSEP to undersand. Leave in some placeholders
so that in future table can be labeld as the proper number.
\begin{quote}\begin{description}
\item[{Parameters}] \leavevmode\begin{itemize}
\item {} 
\sphinxAtStartPar
\sphinxstyleliteralstrong{\sphinxupquote{LogR}} (\sphinxstyleliteralemphasis{\sphinxupquote{ndarray}}) \textendash{} The Log R value array (horizontal axis of table)

\item {} 
\sphinxAtStartPar
\sphinxstyleliteralstrong{\sphinxupquote{LogT}} (\sphinxstyleliteralemphasis{\sphinxupquote{ndarray}}) \textendash{} The Log Temperature value array (vertical axis)

\item {} 
\sphinxAtStartPar
\sphinxstyleliteralstrong{\sphinxupquote{LogRMO}} (\sphinxstyleliteralemphasis{\sphinxupquote{ndarray}}) \textendash{} all of the RMO values associated with R and T

\item {} 
\sphinxAtStartPar
\sphinxstyleliteralstrong{\sphinxupquote{TNUM}} (\sphinxstyleliteralemphasis{\sphinxupquote{int}}) \textendash{} Table number

\item {} 
\sphinxAtStartPar
\sphinxstyleliteralstrong{\sphinxupquote{comp}} (\sphinxstyleliteralemphasis{\sphinxupquote{dict}}\sphinxstyleliteralemphasis{\sphinxupquote{, }}\sphinxstyleliteralemphasis{\sphinxupquote{optional}}) \textendash{} composition dictionary. If not provided placeholders are left in
place so that it may be filled later on

\item {} 
\sphinxAtStartPar
\sphinxstyleliteralstrong{\sphinxupquote{upperNonRect}} (\sphinxstyleliteralemphasis{\sphinxupquote{ndarray}}\sphinxstyleliteralemphasis{\sphinxupquote{, }}\sphinxstyleliteralemphasis{\sphinxupquote{default=None}}) \textendash{} Array describing how to fill the top of the table non rectangurally
This array should be of the shape (nRowsPerTable * nTables, 3). So
if you have 5 tables each with 70 rows then this array should have
a shape of (350,3). The first column of this array correspond to
the table that the row is a member of, the second column correspond
to the row in that table that the row is. so the first row of the
first table would be at upperNonRect{[}0,:{]} = {[}0,0,…{]} while the
first row of the second table would be at upperNonRect{[}n,:{]}
{[}1,0,…{]}. The final column desribes how many of the elements,
counting from the left of the opacity table should be blanked out
to 99.999 (the sentinal value DSEP uses for non entries). So a row
of {[}2,55,8{]} would mean that for the 56th row in the 3rd table blank
out the firts 8 opacity values (opacities for the first 8 values of
logR).

\item {} 
\sphinxAtStartPar
\sphinxstyleliteralstrong{\sphinxupquote{nonRect}} (\sphinxstyleliteralemphasis{\sphinxupquote{bool}}\sphinxstyleliteralemphasis{\sphinxupquote{, }}\sphinxstyleliteralemphasis{\sphinxupquote{default = False}}) \textendash{} Flag to control whether output tables will be rectangular or have
their corners cut off in a way consistant which how DSEP expects
OPAL tables.

\end{itemize}

\item[{Returns}] \leavevmode
\sphinxAtStartPar
\begin{itemize}
\item {} 
\sphinxAtStartPar
\sphinxstylestrong{metaLine} (\sphinxstyleemphasis{str}) \textendash{} header line for each table, may or may not have placeholders in it

\item {} 
\sphinxAtStartPar
\sphinxstylestrong{fullTable} (\sphinxstyleemphasis{str}) \textendash{} full table to be places in opacity file

\end{itemize}


\end{description}\end{quote}

\end{fulllineitems}

\index{parse\_RMO\_TOPS\_table\_file() (in module pyTOPSScrape.api.convert)@\spxentry{parse\_RMO\_TOPS\_table\_file()}\spxextra{in module pyTOPSScrape.api.convert}}

\begin{fulllineitems}
\phantomsection\label{\detokenize{pyTOPSScrape.api:pyTOPSScrape.api.convert.parse_RMO_TOPS_table_file}}\pysiglinewithargsret{\sphinxcode{\sphinxupquote{pyTOPSScrape.api.convert.}}\sphinxbfcode{\sphinxupquote{parse\_RMO\_TOPS\_table\_file}}}{\emph{\DUrole{n}{TOPSTable}\DUrole{p}{:} \DUrole{n}{str}}, \emph{\DUrole{n}{n}\DUrole{p}{:} \DUrole{n}{int} \DUrole{o}{=} \DUrole{default_value}{100}}}{{ $\rightarrow$ Tuple\DUrole{p}{{[}}numpy.ndarray\DUrole{p}{, }numpy.ndarray\DUrole{p}{, }numpy.ndarray\DUrole{p}{{]}}}}
\sphinxAtStartPar
Given the path to a file queried from the TOPS webform put it into a
computer usable form of 3 arrays. One array of mass density, one of
LogT and one of log Rossland Mean Opacity
\begin{quote}\begin{description}
\item[{Parameters}] \leavevmode\begin{itemize}
\item {} 
\sphinxAtStartPar
\sphinxstyleliteralstrong{\sphinxupquote{TOPSTable}} (\sphinxstyleliteralemphasis{\sphinxupquote{string}}) \textendash{} Path to file queried from TOPS webform

\item {} 
\sphinxAtStartPar
\sphinxstyleliteralstrong{\sphinxupquote{n}} (\sphinxstyleliteralemphasis{\sphinxupquote{int}}\sphinxstyleliteralemphasis{\sphinxupquote{, }}\sphinxstyleliteralemphasis{\sphinxupquote{default=100}}) \textendash{} The size of density grid used in TOPS query form.

\end{itemize}

\item[{Returns}] \leavevmode
\sphinxAtStartPar
\begin{itemize}
\item {} 
\sphinxAtStartPar
\sphinxstylestrong{rho} (\sphinxstyleemphasis{np.ndarray(shape=n)}) \textendash{} Array of mass densities (in cgs) parsed from TOPS table.

\item {} 
\sphinxAtStartPar
\sphinxstylestrong{LogT} (\sphinxstyleemphasis{np.ndarray(shape=m)}) \textendash{} Array of temperatures (in Kelvin) parsed from TOPS table.

\item {} 
\sphinxAtStartPar
\sphinxstylestrong{OPALTableInit} (\sphinxstyleemphasis{np.ndarray(shape=(m,n))}) \textendash{} Array of Rossland Mean Opacities parsed from TOPS table.

\end{itemize}


\end{description}\end{quote}

\end{fulllineitems}

\index{rebuild\_formated\_tables() (in module pyTOPSScrape.api.convert)@\spxentry{rebuild\_formated\_tables()}\spxextra{in module pyTOPSScrape.api.convert}}

\begin{fulllineitems}
\phantomsection\label{\detokenize{pyTOPSScrape.api:pyTOPSScrape.api.convert.rebuild_formated_tables}}\pysiglinewithargsret{\sphinxcode{\sphinxupquote{pyTOPSScrape.api.convert.}}\sphinxbfcode{\sphinxupquote{rebuild\_formated\_tables}}}{\emph{\DUrole{n}{formatedTables}}, \emph{\DUrole{n}{interpRMO}}, \emph{\DUrole{n}{pContents}}, \emph{\DUrole{n}{upperNonRect}}, \emph{\DUrole{n}{nonRect}\DUrole{o}{=}\DUrole{default_value}{False}}}{}
\sphinxAtStartPar
Iterate over a list of opacity tables and a list of desired chemical
compositions then replace the contents of the table list with the newly
updated RMOs frmo the interpolation.
\begin{quote}\begin{description}
\item[{Parameters}] \leavevmode\begin{itemize}
\item {} 
\sphinxAtStartPar
\sphinxstyleliteralstrong{\sphinxupquote{formatedTables}} (\sphinxstyleliteralemphasis{\sphinxupquote{list of dicts}}) \textendash{} List of dictionaries holding three axis each. X, Z, and LogRMO.
These are the “observed” values to be interpolared

\item {} 
\sphinxAtStartPar
\sphinxstyleliteralstrong{\sphinxupquote{interpRMO}} (\sphinxstyleliteralemphasis{\sphinxupquote{list of ndarrays}}) \textendash{} RMOs after interpolation to LogR\sphinxhyphen{}LogT space from rho\sphinxhyphen{}LogT space.

\item {} 
\sphinxAtStartPar
\sphinxstyleliteralstrong{\sphinxupquote{pContents}} (\sphinxstyleliteralemphasis{\sphinxupquote{np.array}}\sphinxstyleliteralemphasis{\sphinxupquote{(}}\sphinxstyleliteralemphasis{\sphinxupquote{shape=}}\sphinxstyleliteralemphasis{\sphinxupquote{(}}\sphinxstyleliteralemphasis{\sphinxupquote{n}}\sphinxstyleliteralemphasis{\sphinxupquote{, }}\sphinxstyleliteralemphasis{\sphinxupquote{3}}\sphinxstyleliteralemphasis{\sphinxupquote{)}}\sphinxstyleliteralemphasis{\sphinxupquote{)}}) \textendash{} Numpy array of all the compositions of length n.  For a dsep n=126.
Along the second axis the first column is X, the second is Y, and
the third is Z.

\item {} 
\sphinxAtStartPar
\sphinxstyleliteralstrong{\sphinxupquote{upperNonRect}} (\sphinxstyleliteralemphasis{\sphinxupquote{ndarray}}) \textendash{} Array describing how to fill the top of the table non rectangurally
This array should be of the shape (nRowsPerTable * nTables, 3). So
if you have 5 tables each with 70 rows then this array should have
a shape of (350,3). The first column of this array correspond to
the table that the row is a member of, the second column correspond
to the row in that table that the row is. so the first row of the
first table would be at upperNonRect{[}0,:{]} = {[}0,0,…{]} while the
first row of the second table would be at upperNonRect{[}n,:{]}
{[}1,0,…{]}. The final column desribes how many of the elements,
counting from the left of the opacity table should be blanked out
to 99.999 (the sentinal value DSEP uses for non entries). So a row
of {[}2,55,8{]} would mean that for the 56th row in the 3rd table blank
out the firts 8 opacity values (opacities for the first 8 values of
logR).

\item {} 
\sphinxAtStartPar
\sphinxstyleliteralstrong{\sphinxupquote{nonRect}} (\sphinxstyleliteralemphasis{\sphinxupquote{bool}}\sphinxstyleliteralemphasis{\sphinxupquote{, }}\sphinxstyleliteralemphasis{\sphinxupquote{default = False}}) \textendash{} Flag to control whether output tables will be rectangular or have
their corners cut off in a way consistant which how DSEP expects
OPAL tables.

\end{itemize}

\item[{Returns}] \leavevmode
\sphinxAtStartPar
\sphinxstylestrong{formatedTables} \textendash{} List of dictionaries holding three axis each. X, Z, and LogRMO.
These have been updated to reflect the compositions in pContents.

\item[{Return type}] \leavevmode
\sphinxAtStartPar
list of dicts

\end{description}\end{quote}

\end{fulllineitems}



\section{pyTOPSScrape.api.utils module}
\label{\detokenize{pyTOPSScrape.api:module-pyTOPSScrape.api.utils}}\label{\detokenize{pyTOPSScrape.api:pytopsscrape-api-utils-module}}\index{module@\spxentry{module}!pyTOPSScrape.api.utils@\spxentry{pyTOPSScrape.api.utils}}\index{pyTOPSScrape.api.utils@\spxentry{pyTOPSScrape.api.utils}!module@\spxentry{module}}
\sphinxAtStartPar
\sphinxstylestrong{Author:} Thomas M. Boudreaux

\sphinxAtStartPar
\sphinxstylestrong{Created:} September 2021

\sphinxAtStartPar
\sphinxstylestrong{Last Modified:} September 2021

\sphinxAtStartPar
Utilities to help with the TOPS query api
\index{format\_TOPS\_string() (in module pyTOPSScrape.api.utils)@\spxentry{format\_TOPS\_string()}\spxextra{in module pyTOPSScrape.api.utils}}

\begin{fulllineitems}
\phantomsection\label{\detokenize{pyTOPSScrape.api:pyTOPSScrape.api.utils.format_TOPS_string}}\pysiglinewithargsret{\sphinxcode{\sphinxupquote{pyTOPSScrape.api.utils.}}\sphinxbfcode{\sphinxupquote{format\_TOPS\_string}}}{\emph{\DUrole{n}{compList}\DUrole{p}{:} \DUrole{n}{list}}}{{ $\rightarrow$ str}}
\sphinxAtStartPar
Format the composition list from pasrse\_abundance\_file into a string in the
form that the TOPS web form expects for a mass fraction input.
\begin{quote}\begin{description}
\item[{Parameters}] \leavevmode
\sphinxAtStartPar
\sphinxstyleliteralstrong{\sphinxupquote{compList}} (\sphinxstyleliteralemphasis{\sphinxupquote{list}}) \textendash{} composition list in the form of: {[}(‘Element’, massFrac,
numFrac),…{]}

\item[{Returns}] \leavevmode
\sphinxAtStartPar
\sphinxstylestrong{TOPS\_abundance\_string} \textendash{} string in the form of: “massFrac0 Element0 massFrac1 Element1 …”

\item[{Return type}] \leavevmode
\sphinxAtStartPar
string

\end{description}\end{quote}

\end{fulllineitems}

\index{validate\_extant\_tables() (in module pyTOPSScrape.api.utils)@\spxentry{validate\_extant\_tables()}\spxextra{in module pyTOPSScrape.api.utils}}

\begin{fulllineitems}
\phantomsection\label{\detokenize{pyTOPSScrape.api:pyTOPSScrape.api.utils.validate_extant_tables}}\pysiglinewithargsret{\sphinxcode{\sphinxupquote{pyTOPSScrape.api.utils.}}\sphinxbfcode{\sphinxupquote{validate\_extant\_tables}}}{\emph{\DUrole{n}{path}\DUrole{p}{:} \DUrole{n}{str}}, \emph{\DUrole{n}{prefix}\DUrole{p}{:} \DUrole{n}{str}}}{{ $\rightarrow$ bool}}
\sphinxAtStartPar
Check if there is a quiried table from TOPS for every number frac file
generated by the program passed to call\_num\_frac.
\begin{quote}\begin{description}
\item[{Parameters}] \leavevmode\begin{itemize}
\item {} 
\sphinxAtStartPar
\sphinxstyleliteralstrong{\sphinxupquote{path}} (\sphinxstyleliteralemphasis{\sphinxupquote{string}}) \textendash{} Path to where the results of the number frac and TOPS query files
are stored

\item {} 
\sphinxAtStartPar
\sphinxstyleliteralstrong{\sphinxupquote{prefix}} (\sphinxstyleliteralemphasis{\sphinxupquote{string}}) \textendash{} start prefix given to all abundance / number frac files

\end{itemize}

\item[{Returns}] \leavevmode
\sphinxAtStartPar
\sphinxstylestrong{validated} \textendash{} Whether or not all number frac files have a corresponding TOPS
opacity table

\item[{Return type}] \leavevmode
\sphinxAtStartPar
bool

\end{description}\end{quote}

\end{fulllineitems}



\section{Module contents}
\label{\detokenize{pyTOPSScrape.api:module-pyTOPSScrape.api}}\label{\detokenize{pyTOPSScrape.api:module-contents}}\index{module@\spxentry{module}!pyTOPSScrape.api@\spxentry{pyTOPSScrape.api}}\index{pyTOPSScrape.api@\spxentry{pyTOPSScrape.api}!module@\spxentry{module}}

\chapter{pyTOPSScrape.err package}
\label{\detokenize{pyTOPSScrape.err:pytopsscrape-err-package}}\label{\detokenize{pyTOPSScrape.err::doc}}

\section{Submodules}
\label{\detokenize{pyTOPSScrape.err:submodules}}

\section{pyTOPSScrape.err.err module}
\label{\detokenize{pyTOPSScrape.err:module-pyTOPSScrape.err.err}}\label{\detokenize{pyTOPSScrape.err:pytopsscrape-err-err-module}}\index{module@\spxentry{module}!pyTOPSScrape.err.err@\spxentry{pyTOPSScrape.err.err}}\index{pyTOPSScrape.err.err@\spxentry{pyTOPSScrape.err.err}!module@\spxentry{module}}

\section{Module contents}
\label{\detokenize{pyTOPSScrape.err:module-pyTOPSScrape.err}}\label{\detokenize{pyTOPSScrape.err:module-contents}}\index{module@\spxentry{module}!pyTOPSScrape.err@\spxentry{pyTOPSScrape.err}}\index{pyTOPSScrape.err@\spxentry{pyTOPSScrape.err}!module@\spxentry{module}}

\chapter{pyTOPSScrape.ext package}
\label{\detokenize{pyTOPSScrape.ext:pytopsscrape-ext-package}}\label{\detokenize{pyTOPSScrape.ext::doc}}

\section{Submodules}
\label{\detokenize{pyTOPSScrape.ext:submodules}}

\section{pyTOPSScrape.ext.utils module}
\label{\detokenize{pyTOPSScrape.ext:module-pyTOPSScrape.ext.utils}}\label{\detokenize{pyTOPSScrape.ext:pytopsscrape-ext-utils-module}}\index{module@\spxentry{module}!pyTOPSScrape.ext.utils@\spxentry{pyTOPSScrape.ext.utils}}\index{pyTOPSScrape.ext.utils@\spxentry{pyTOPSScrape.ext.utils}!module@\spxentry{module}}\index{call\_num\_frac() (in module pyTOPSScrape.ext.utils)@\spxentry{call\_num\_frac()}\spxextra{in module pyTOPSScrape.ext.utils}}

\begin{fulllineitems}
\phantomsection\label{\detokenize{pyTOPSScrape.ext:pyTOPSScrape.ext.utils.call_num_frac}}\pysiglinewithargsret{\sphinxcode{\sphinxupquote{pyTOPSScrape.ext.utils.}}\sphinxbfcode{\sphinxupquote{call\_num\_frac}}}{\emph{\DUrole{n}{abunTable}\DUrole{p}{:} \DUrole{n}{str}}, \emph{\DUrole{n}{feh}\DUrole{p}{:} \DUrole{n}{Union\DUrole{p}{{[}}float\DUrole{p}{, }Tuple\DUrole{p}{{[}}float\DUrole{p}{, }float\DUrole{p}{, }int\DUrole{p}{{]}}\DUrole{p}{{]}}}}, \emph{\DUrole{n}{alpha}\DUrole{p}{:} \DUrole{n}{Union\DUrole{p}{{[}}float\DUrole{p}{, }Tuple\DUrole{p}{{[}}float\DUrole{p}{, }float\DUrole{p}{, }int\DUrole{p}{{]}}\DUrole{p}{{]}}}}, \emph{\DUrole{n}{Y}\DUrole{p}{:} \DUrole{n}{Union\DUrole{p}{{[}}float\DUrole{p}{, }Tuple\DUrole{p}{{[}}float\DUrole{p}{, }float\DUrole{p}{, }int\DUrole{p}{{]}}\DUrole{p}{{]}}}}, \emph{\DUrole{n}{Xc}\DUrole{p}{:} \DUrole{n}{float}}, \emph{\DUrole{n}{Yc}\DUrole{p}{:} \DUrole{n}{float}}}{{ $\rightarrow$ \_io.BytesIO}}
\sphinxAtStartPar
Given some grid of {[}Fe/H{]}, {[}Alpha/Fe{]}, and a(He) generate the number
fractions files for every point on that grid. This is done using the
libnumfrac shared library in ext/lib.
\begin{quote}\begin{description}
\item[{Parameters}] \leavevmode\begin{itemize}
\item {} 
\sphinxAtStartPar
\sphinxstyleliteralstrong{\sphinxupquote{abunTable}} (\sphinxstyleliteralemphasis{\sphinxupquote{string}}) \textendash{} Table to parse abundances from.

\item {} 
\sphinxAtStartPar
\sphinxstyleliteralstrong{\sphinxupquote{feh}} (\sphinxstyleliteralemphasis{\sphinxupquote{float}}\sphinxstyleliteralemphasis{\sphinxupquote{ or }}\sphinxstyleliteralemphasis{\sphinxupquote{tuple of floats}}) \textendash{} {[}Fe/H{]} value to pass to program. If this is a float the numFrac
program will only be evaluate at that point, if a tuple it will be
evaluated at every point within linspace(feh{[}0{]}, feh{[}1{]}, feh{[}2{]}).

\item {} 
\sphinxAtStartPar
\sphinxstyleliteralstrong{\sphinxupquote{alpha}} (\sphinxstyleliteralemphasis{\sphinxupquote{float of tuple of floats}}) \textendash{} {[}alpha/H{]} value to pass to program. If this is a float the numFrac
program will only be evaluate at that point, if a tuple it will be
evaluated at every point within linspace(alpha{[}0{]}, alpha{[}1{]},
alpha{[}2{]}).

\item {} 
\sphinxAtStartPar
\sphinxstyleliteralstrong{\sphinxupquote{Y}} (\sphinxstyleliteralemphasis{\sphinxupquote{float}}\sphinxstyleliteralemphasis{\sphinxupquote{ or }}\sphinxstyleliteralemphasis{\sphinxupquote{tuple of floats}}) \textendash{} a(He) value to pass to program. If this is a float the numFrac
program will only be evaluate at that point, if a tuple it will be
evaluated at every point within linspace(Y{[}0{]}, Y{[}1{]}, Y{[}2{]}).

\item {} 
\sphinxAtStartPar
\sphinxstyleliteralstrong{\sphinxupquote{Xc}} (\sphinxstyleliteralemphasis{\sphinxupquote{float}}) \textendash{} Current X to use as a reference

\item {} 
\sphinxAtStartPar
\sphinxstyleliteralstrong{\sphinxupquote{Yc}} (\sphinxstyleliteralemphasis{\sphinxupquote{float}}) \textendash{} Current Y to use as a referece (only used if Xc == 0)

\end{itemize}

\item[{Returns}] \leavevmode
\sphinxAtStartPar
\sphinxstylestrong{fp} \textendash{} Temporary file object storing results

\item[{Return type}] \leavevmode
\sphinxAtStartPar
BytesIO

\end{description}\end{quote}

\end{fulllineitems}

\index{get\_base\_composition() (in module pyTOPSScrape.ext.utils)@\spxentry{get\_base\_composition()}\spxextra{in module pyTOPSScrape.ext.utils}}

\begin{fulllineitems}
\phantomsection\label{\detokenize{pyTOPSScrape.ext:pyTOPSScrape.ext.utils.get_base_composition}}\pysiglinewithargsret{\sphinxcode{\sphinxupquote{pyTOPSScrape.ext.utils.}}\sphinxbfcode{\sphinxupquote{get\_base\_composition}}}{\emph{\DUrole{n}{aTablePath}\DUrole{p}{:} \DUrole{n}{str}}}{{ $\rightarrow$ Tuple\DUrole{p}{{[}}list\DUrole{p}{, }float\DUrole{p}{, }float\DUrole{p}{, }float\DUrole{p}{{]}}}}
\sphinxAtStartPar
For some abundance path return the “base” composition, this is mainly to be
used for headers.
\begin{quote}\begin{description}
\item[{Parameters}] \leavevmode
\sphinxAtStartPar
\sphinxstyleliteralstrong{\sphinxupquote{aTablePath}} (\sphinxstyleliteralemphasis{\sphinxupquote{str}}) \textendash{} Path to the abundance table in the form as described in the
parseChemFile module documentation

\item[{Returns}] \leavevmode
\sphinxAtStartPar
\begin{itemize}
\item {} 
\sphinxAtStartPar
\sphinxstyleemphasis{list} \textendash{} list of the composition in the form {[}(‘Element’,massFrac,numberFrac),…{]}

\item {} 
\sphinxAtStartPar
\sphinxstyleemphasis{float} \textendash{} Hydrogen mass fraction

\item {} 
\sphinxAtStartPar
\sphinxstyleemphasis{float} \textendash{} Helium mass fraction

\item {} 
\sphinxAtStartPar
\sphinxstyleemphasis{float} \textendash{} Metal mass fraction

\end{itemize}


\end{description}\end{quote}

\end{fulllineitems}

\index{parse\_numfrac\_file() (in module pyTOPSScrape.ext.utils)@\spxentry{parse\_numfrac\_file()}\spxextra{in module pyTOPSScrape.ext.utils}}

\begin{fulllineitems}
\phantomsection\label{\detokenize{pyTOPSScrape.ext:pyTOPSScrape.ext.utils.parse_numfrac_file}}\pysiglinewithargsret{\sphinxcode{\sphinxupquote{pyTOPSScrape.ext.utils.}}\sphinxbfcode{\sphinxupquote{parse\_numfrac\_file}}}{\emph{\DUrole{n}{file}\DUrole{p}{:} \DUrole{n}{\_io.BytesIO}}, \emph{\DUrole{n}{big}\DUrole{p}{:} \DUrole{n}{bool} \DUrole{o}{=} \DUrole{default_value}{False}}, \emph{\DUrole{n}{pbar}\DUrole{p}{:} \DUrole{n}{bool} \DUrole{o}{=} \DUrole{default_value}{True}}}{{ $\rightarrow$ Tuple\DUrole{p}{{[}}numpy.ndarray\DUrole{p}{, }float\DUrole{p}{, }float\DUrole{p}{, }float\DUrole{p}{{]}}}}
\sphinxAtStartPar
Given a file generated by the executable used in call\_num\_frac parse that
file into a usable form. This includes the hydrogen, helium, and
metall mass fractions. And a list in the form of
{[}(‘Element’, massFrac, numberFrac),…{]}
\begin{quote}\begin{description}
\item[{Parameters}] \leavevmode\begin{itemize}
\item {} 
\sphinxAtStartPar
\sphinxstyleliteralstrong{\sphinxupquote{file}} (\sphinxstyleliteralemphasis{\sphinxupquote{BytesIO}}) \textendash{} file like object to abundance file

\item {} 
\sphinxAtStartPar
\sphinxstyleliteralstrong{\sphinxupquote{big}} (\sphinxstyleliteralemphasis{\sphinxupquote{bool}}) \textendash{} single composition file or one file with many composition

\item {} 
\sphinxAtStartPar
\sphinxstyleliteralstrong{\sphinxupquote{pbad}} (\sphinxstyleliteralemphasis{\sphinxupquote{bool}}) \textendash{} display progress bar

\end{itemize}

\item[{Returns}] \leavevmode
\sphinxAtStartPar
\begin{itemize}
\item {} 
\sphinxAtStartPar
\sphinxstyleemphasis{list} \textendash{} list of the composition in the form {[}(‘Element’,massFrac,numberFrac),…{]}

\item {} 
\sphinxAtStartPar
\sphinxstyleemphasis{float} \textendash{} Hydrogen mass fraction

\item {} 
\sphinxAtStartPar
\sphinxstyleemphasis{float} \textendash{} Helium mass fraction

\item {} 
\sphinxAtStartPar
\sphinxstyleemphasis{float} \textendash{} Metal mass fraction

\end{itemize}


\end{description}\end{quote}

\end{fulllineitems}



\section{Module contents}
\label{\detokenize{pyTOPSScrape.ext:module-pyTOPSScrape.ext}}\label{\detokenize{pyTOPSScrape.ext:module-contents}}\index{module@\spxentry{module}!pyTOPSScrape.ext@\spxentry{pyTOPSScrape.ext}}\index{pyTOPSScrape.ext@\spxentry{pyTOPSScrape.ext}!module@\spxentry{module}}

\chapter{pyTOPSScrape.parse package}
\label{\detokenize{pyTOPSScrape.parse:pytopsscrape-parse-package}}\label{\detokenize{pyTOPSScrape.parse::doc}}

\section{Submodules}
\label{\detokenize{pyTOPSScrape.parse:submodules}}

\section{pyTOPSScrape.parse.abundance module}
\label{\detokenize{pyTOPSScrape.parse:module-pyTOPSScrape.parse.abundance}}\label{\detokenize{pyTOPSScrape.parse:pytopsscrape-parse-abundance-module}}\index{module@\spxentry{module}!pyTOPSScrape.parse.abundance@\spxentry{pyTOPSScrape.parse.abundance}}\index{pyTOPSScrape.parse.abundance@\spxentry{pyTOPSScrape.parse.abundance}!module@\spxentry{module}}
\sphinxAtStartPar
\sphinxstylestrong{Author:} Thomas M. Boudreaux

\sphinxAtStartPar
\sphinxstylestrong{Created:} May 2021

\sphinxAtStartPar
\sphinxstylestrong{Last Modified:} May 2021

\sphinxAtStartPar
Module responsible for the parsing and handeling of chemical composition files
in the form of

\begin{sphinxVerbatim}[commandchars=\\\{\}]
\PYG{c+c1}{\PYGZsh{}STD [Fe/H] [alpha/Fe] [C/Fe] [N/Fe] [O/Fe] [r/Fe] [s/Fe] C/O X Y,Z}
\PYG{n}{F} \PYG{o}{\PYGZhy{}}\PYG{l+m+mf}{1.13} \PYG{l+m+mf}{0.32} \PYG{o}{\PYGZhy{}}\PYG{l+m+mf}{0.43} \PYG{o}{\PYGZhy{}}\PYG{l+m+mf}{0.28} \PYG{l+m+mf}{0.31} \PYG{o}{\PYGZhy{}}\PYG{l+m+mf}{1.13} \PYG{o}{\PYGZhy{}}\PYG{l+m+mf}{1.13} \PYG{l+m+mf}{0.10} \PYG{l+m+mf}{0.7584} \PYG{l+m+mf}{0.2400}\PYG{p}{,}\PYG{l+m+mf}{1.599E\PYGZhy{}03}
\PYG{c+c1}{\PYGZsh{}H He Li Be B C N O F Ne}
\PYG{l+m+mf}{12.00} \PYG{l+m+mf}{10.898} \PYG{o}{\PYGZhy{}}\PYG{l+m+mf}{0.08} \PYG{l+m+mf}{0.25} \PYG{l+m+mf}{1.57} \PYG{l+m+mf}{6.87} \PYG{l+m+mf}{6.42} \PYG{l+m+mf}{7.87} \PYG{l+m+mf}{3.43} \PYG{l+m+mf}{7.12}
\PYG{c+c1}{\PYGZsh{}Na Mg Al Si P S Cl Ar K Ca}
\PYG{l+m+mf}{5.11} \PYG{l+m+mf}{6.86} \PYG{l+m+mf}{5.21} \PYG{l+m+mf}{6.65} \PYG{l+m+mf}{4.28} \PYG{l+m+mf}{6.31} \PYG{o}{\PYGZhy{}}\PYG{l+m+mf}{1.13} \PYG{l+m+mf}{5.59} \PYG{l+m+mf}{3.90} \PYG{l+m+mf}{5.21}
\PYG{c+c1}{\PYGZsh{}Sc Ti V Cr Mn Fe Co Ni Cu Zn}
\PYG{l+m+mf}{2.02} \PYG{l+m+mf}{3.82} \PYG{l+m+mf}{2.80} \PYG{l+m+mf}{4.51} \PYG{l+m+mf}{4.30} \PYG{l+m+mf}{6.37} \PYG{l+m+mf}{3.86} \PYG{l+m+mf}{5.09} \PYG{l+m+mf}{3.06} \PYG{l+m+mf}{2.30}
\PYG{c+c1}{\PYGZsh{}Ga Ge As Se Br Kr Rb Sr Y Zr}
\PYG{l+m+mf}{0.78} \PYG{l+m+mf}{1.39} \PYG{l+m+mf}{0.04} \PYG{l+m+mf}{1.08} \PYG{l+m+mf}{0.28} \PYG{l+m+mf}{0.99} \PYG{l+m+mf}{0.26} \PYG{l+m+mf}{0.61} \PYG{l+m+mf}{1.08} \PYG{l+m+mf}{1.45}
\PYG{c+c1}{\PYGZsh{}Nb Mo Tc Ru Rh Pd Ag Cd In Sn}
\PYG{o}{\PYGZhy{}}\PYG{l+m+mf}{0.80} \PYG{o}{\PYGZhy{}}\PYG{l+m+mf}{0.38} \PYG{o}{\PYGZhy{}}\PYG{l+m+mf}{99.00} \PYG{o}{\PYGZhy{}}\PYG{l+m+mf}{0.51} \PYG{o}{\PYGZhy{}}\PYG{l+m+mf}{1.35} \PYG{o}{\PYGZhy{}}\PYG{l+m+mf}{0.69} \PYG{o}{\PYGZhy{}}\PYG{l+m+mf}{1.32} \PYG{o}{\PYGZhy{}}\PYG{l+m+mf}{0.55} \PYG{o}{\PYGZhy{}}\PYG{l+m+mf}{1.46} \PYG{o}{\PYGZhy{}}\PYG{l+m+mf}{0.22}
\PYG{c+c1}{\PYGZsh{}Sb Te I Xe Cs Ba La Ce Pr Nd}
\PYG{o}{\PYGZhy{}}\PYG{l+m+mf}{1.25} \PYG{o}{\PYGZhy{}}\PYG{l+m+mf}{0.08} \PYG{o}{\PYGZhy{}}\PYG{l+m+mf}{0.71} \PYG{o}{\PYGZhy{}}\PYG{l+m+mf}{0.02} \PYG{o}{\PYGZhy{}}\PYG{l+m+mf}{1.18} \PYG{l+m+mf}{1.05} \PYG{o}{\PYGZhy{}}\PYG{l+m+mf}{0.03} \PYG{l+m+mf}{0.45} \PYG{o}{\PYGZhy{}}\PYG{l+m+mf}{1.54} \PYG{l+m+mf}{0.29}
\PYG{c+c1}{\PYGZsh{}Pm Sm Eu Gd Tb Dy Ho Er Tm Yb}
\PYG{o}{\PYGZhy{}}\PYG{l+m+mf}{99.00} \PYG{o}{\PYGZhy{}}\PYG{l+m+mf}{1.30} \PYG{o}{\PYGZhy{}}\PYG{l+m+mf}{0.61} \PYG{o}{\PYGZhy{}}\PYG{l+m+mf}{1.19} \PYG{o}{\PYGZhy{}}\PYG{l+m+mf}{1.96} \PYG{o}{\PYGZhy{}}\PYG{l+m+mf}{1.16} \PYG{o}{\PYGZhy{}}\PYG{l+m+mf}{1.78} \PYG{o}{\PYGZhy{}}\PYG{l+m+mf}{1.34} \PYG{o}{\PYGZhy{}}\PYG{l+m+mf}{2.16} \PYG{o}{\PYGZhy{}}\PYG{l+m+mf}{1.42}
\PYG{c+c1}{\PYGZsh{}Lu Hf Ta W Re Os Ir Pt Au Hg}
\PYG{o}{\PYGZhy{}}\PYG{l+m+mf}{2.16} \PYG{o}{\PYGZhy{}}\PYG{l+m+mf}{1.41} \PYG{o}{\PYGZhy{}}\PYG{l+m+mf}{2.38} \PYG{o}{\PYGZhy{}}\PYG{l+m+mf}{1.41} \PYG{o}{\PYGZhy{}}\PYG{l+m+mf}{2.00} \PYG{o}{\PYGZhy{}}\PYG{l+m+mf}{0.86} \PYG{o}{\PYGZhy{}}\PYG{l+m+mf}{0.88} \PYG{o}{\PYGZhy{}}\PYG{l+m+mf}{0.64} \PYG{o}{\PYGZhy{}}\PYG{l+m+mf}{1.34} \PYG{o}{\PYGZhy{}}\PYG{l+m+mf}{1.09}
\PYG{c+c1}{\PYGZsh{}Tl Pb Bi Po At Rn Fr Ra Ac Th}
\PYG{o}{\PYGZhy{}}\PYG{l+m+mf}{1.36} \PYG{o}{\PYGZhy{}}\PYG{l+m+mf}{0.51} \PYG{o}{\PYGZhy{}}\PYG{l+m+mf}{1.61} \PYG{o}{\PYGZhy{}}\PYG{l+m+mf}{99.00} \PYG{o}{\PYGZhy{}}\PYG{l+m+mf}{99.00} \PYG{o}{\PYGZhy{}}\PYG{l+m+mf}{99.00} \PYG{o}{\PYGZhy{}}\PYG{l+m+mf}{99.00} \PYG{o}{\PYGZhy{}}\PYG{l+m+mf}{99.00} \PYG{o}{\PYGZhy{}}\PYG{l+m+mf}{99.00} \PYG{o}{\PYGZhy{}}\PYG{l+m+mf}{2.20}
\PYG{c+c1}{\PYGZsh{}Pa U}
\PYG{o}{\PYGZhy{}}\PYG{l+m+mf}{99.00} \PYG{o}{\PYGZhy{}}\PYG{l+m+mf}{2.80}
\end{sphinxVerbatim}

\sphinxAtStartPar
Where each number is a(i) for the ith element and lines starting with \# are
comments.
\index{a\_to\_mfrac() (in module pyTOPSScrape.parse.abundance)@\spxentry{a\_to\_mfrac()}\spxextra{in module pyTOPSScrape.parse.abundance}}

\begin{fulllineitems}
\phantomsection\label{\detokenize{pyTOPSScrape.parse:pyTOPSScrape.parse.abundance.a_to_mfrac}}\pysiglinewithargsret{\sphinxcode{\sphinxupquote{pyTOPSScrape.parse.abundance.}}\sphinxbfcode{\sphinxupquote{a\_to\_mfrac}}}{\emph{\DUrole{n}{a}}, \emph{\DUrole{n}{amass}}, \emph{\DUrole{n}{X}}}{}
\sphinxAtStartPar
Convert \(a(i)\) for the \(i^{th}\) element to a mass fraction using the expression
\begin{equation*}
\begin{split}a(i) = \log(1.008) + \log(F_{i}) - \left[\log(X) + \log(m_{i})\right] + 12\end{split}
\end{equation*}
\sphinxAtStartPar
Or, equivilenetly, to go from \(a(i)\) to mass fraction
\begin{equation*}
\begin{split}F_{i} = \left[\frac{X m_{i}}{1.008}\right]\times 10^{a(i)-12}\end{split}
\end{equation*}
\sphinxAtStartPar
Where \(F_{i}\) is the math fraction of the \(i^{th}\) element,
\(X\) is the Hydrogen mass fraction, and \(m_{i}\) is the ith
element mass in hydrogen masses.
\begin{quote}\begin{description}
\item[{Parameters}] \leavevmode\begin{itemize}
\item {} 
\sphinxAtStartPar
\sphinxstyleliteralstrong{\sphinxupquote{a}} (\sphinxstyleliteralemphasis{\sphinxupquote{float}}) \textendash{} \(a(i)\) for the \(i^{th}\) element. For example for He chem might
be 10.93. For Hydrogen it would definititionally be 12.

\item {} 
\sphinxAtStartPar
\sphinxstyleliteralstrong{\sphinxupquote{amass}} (\sphinxstyleliteralemphasis{\sphinxupquote{float}}) \textendash{} Mass of \(i^{th}\) element given in atomic mass units.

\item {} 
\sphinxAtStartPar
\sphinxstyleliteralstrong{\sphinxupquote{X}} (\sphinxstyleliteralemphasis{\sphinxupquote{float}}) \textendash{} Hydrogen mass fraction

\end{itemize}

\item[{Returns}] \leavevmode
\sphinxAtStartPar
\sphinxstylestrong{mf} \textendash{} Mass fraction of \(i^{th}\) element.

\item[{Return type}] \leavevmode
\sphinxAtStartPar
float

\end{description}\end{quote}

\end{fulllineitems}

\index{est\_feh\_from\_Z\_and\_X() (in module pyTOPSScrape.parse.abundance)@\spxentry{est\_feh\_from\_Z\_and\_X()}\spxextra{in module pyTOPSScrape.parse.abundance}}

\begin{fulllineitems}
\phantomsection\label{\detokenize{pyTOPSScrape.parse:pyTOPSScrape.parse.abundance.est_feh_from_Z_and_X}}\pysiglinewithargsret{\sphinxcode{\sphinxupquote{pyTOPSScrape.parse.abundance.}}\sphinxbfcode{\sphinxupquote{est\_feh\_from\_Z\_and\_X}}}{\emph{\DUrole{n}{abunTable}\DUrole{p}{:} \DUrole{n}{dict}}, \emph{\DUrole{n}{Xt}\DUrole{p}{:} \DUrole{n}{float}}, \emph{\DUrole{n}{Zt}\DUrole{p}{:} \DUrole{n}{float}}}{{ $\rightarrow$ float}}
\sphinxAtStartPar
Analytically estimate feh from Z and X
\begin{quote}\begin{description}
\item[{Parameters}] \leavevmode\begin{itemize}
\item {} 
\sphinxAtStartPar
\sphinxstyleliteralstrong{\sphinxupquote{abunTable}} (\sphinxstyleliteralemphasis{\sphinxupquote{dict}}) \textendash{} Abundance Table dictionary in the form described in the docs for
pysep.misc.abun.util.open\_and\_parse.

\item {} 
\sphinxAtStartPar
\sphinxstyleliteralstrong{\sphinxupquote{Xt}} (\sphinxstyleliteralemphasis{\sphinxupquote{float}}) \textendash{} Target X to move to

\item {} 
\sphinxAtStartPar
\sphinxstyleliteralstrong{\sphinxupquote{Zt}} (\sphinxstyleliteralemphasis{\sphinxupquote{float}}) \textendash{} Target Z to move to.

\end{itemize}

\item[{Returns}] \leavevmode
\sphinxAtStartPar
\sphinxstylestrong{FeH} \textendash{} {[}Fe/H{]} value to add to every a(i) for every tracked element i where
i \textgreater{} 2 (i.e all the metals).

\item[{Return type}] \leavevmode
\sphinxAtStartPar
float

\end{description}\end{quote}

\end{fulllineitems}

\index{gen\_abun\_map() (in module pyTOPSScrape.parse.abundance)@\spxentry{gen\_abun\_map()}\spxextra{in module pyTOPSScrape.parse.abundance}}

\begin{fulllineitems}
\phantomsection\label{\detokenize{pyTOPSScrape.parse:pyTOPSScrape.parse.abundance.gen_abun_map}}\pysiglinewithargsret{\sphinxcode{\sphinxupquote{pyTOPSScrape.parse.abundance.}}\sphinxbfcode{\sphinxupquote{gen\_abun\_map}}}{\emph{\DUrole{n}{abunTable}}}{}
\sphinxAtStartPar
Generate an analytic mapping between X, Y, Z and FeH given an abundance
table.
\begin{quote}\begin{description}
\item[{Parameters}] \leavevmode
\sphinxAtStartPar
\sphinxstyleliteralstrong{\sphinxupquote{abunTable}} (\sphinxstyleliteralemphasis{\sphinxupquote{str}}) \textendash{} Path of checmical abundance table to use for composition. Format of
this table is defined in the ext module documentation.

\item[{Returns}] \leavevmode
\sphinxAtStartPar
\sphinxstylestrong{MetalAbunMap} \textendash{} Function build from interpolation of a grid of FeH, alpha/Fe, and
a(He) which will returned the set of those values giving the
composition most similar to an input X, Y, and Z.

\item[{Return type}] \leavevmode
\sphinxAtStartPar
function(X,Y,Z) \sphinxhyphen{}\textgreater{} (Fe/H,0.0,a(He))

\end{description}\end{quote}

\end{fulllineitems}

\index{get\_atomic\_masses() (in module pyTOPSScrape.parse.abundance)@\spxentry{get\_atomic\_masses()}\spxextra{in module pyTOPSScrape.parse.abundance}}

\begin{fulllineitems}
\phantomsection\label{\detokenize{pyTOPSScrape.parse:pyTOPSScrape.parse.abundance.get_atomic_masses}}\pysiglinewithargsret{\sphinxcode{\sphinxupquote{pyTOPSScrape.parse.abundance.}}\sphinxbfcode{\sphinxupquote{get\_atomic\_masses}}}{}{}
\sphinxAtStartPar
Return a dict of atomic masses from Hydrogen all the way to plutonium
\begin{quote}\begin{description}
\item[{Returns}] \leavevmode
\sphinxAtStartPar
\sphinxstylestrong{amasses} \textendash{} Dicionary of atomic masses in atomic mass units indexed by elemental
symbol.

\item[{Return type}] \leavevmode
\sphinxAtStartPar
dict of floats

\end{description}\end{quote}

\end{fulllineitems}

\index{mfrac\_to\_a() (in module pyTOPSScrape.parse.abundance)@\spxentry{mfrac\_to\_a()}\spxextra{in module pyTOPSScrape.parse.abundance}}

\begin{fulllineitems}
\phantomsection\label{\detokenize{pyTOPSScrape.parse:pyTOPSScrape.parse.abundance.mfrac_to_a}}\pysiglinewithargsret{\sphinxcode{\sphinxupquote{pyTOPSScrape.parse.abundance.}}\sphinxbfcode{\sphinxupquote{mfrac\_to\_a}}}{\emph{\DUrole{n}{mfrac}}, \emph{\DUrole{n}{amass}}, \emph{\DUrole{n}{X}}, \emph{\DUrole{n}{Y}}}{}
\sphinxAtStartPar
Convert mass fracition of a given element to a for that element at a given
hydrogen mass fraction using the equation
\begin{equation*}
\begin{split}a(i) = \log(1.008) + \log(F_{i}) - \left[\log(X) + \log(m_{i}) \right]\end{split}
\end{equation*}
\sphinxAtStartPar
Where \(F_{i}\) is the mass fraction for the \(i^{th}\) element and
\(m_{i}\) is the mass fraction for the \(i^{th}\) element.
\begin{quote}\begin{description}
\item[{Parameters}] \leavevmode\begin{itemize}
\item {} 
\sphinxAtStartPar
\sphinxstyleliteralstrong{\sphinxupquote{mfrac}} (\sphinxstyleliteralemphasis{\sphinxupquote{float}}) \textendash{} Mass fraction of the ith element.

\item {} 
\sphinxAtStartPar
\sphinxstyleliteralstrong{\sphinxupquote{amass}} (\sphinxstyleliteralemphasis{\sphinxupquote{float}}) \textendash{} Mass of the ith element in atomic mass units.

\item {} 
\sphinxAtStartPar
\sphinxstyleliteralstrong{\sphinxupquote{X}} (\sphinxstyleliteralemphasis{\sphinxupquote{float}}) \textendash{} Hydrogen mass fraction

\item {} 
\sphinxAtStartPar
\sphinxstyleliteralstrong{\sphinxupquote{Y}} (\sphinxstyleliteralemphasis{\sphinxupquote{float}}) \textendash{} Helium mass fraction, will be used as reference if X = 0

\end{itemize}

\item[{Returns}] \leavevmode
\sphinxAtStartPar
\sphinxstylestrong{a} \textendash{} a for the ith element

\item[{Return type}] \leavevmode
\sphinxAtStartPar
float

\end{description}\end{quote}

\end{fulllineitems}

\index{open\_and\_parse() (in module pyTOPSScrape.parse.abundance)@\spxentry{open\_and\_parse()}\spxextra{in module pyTOPSScrape.parse.abundance}}

\begin{fulllineitems}
\phantomsection\label{\detokenize{pyTOPSScrape.parse:pyTOPSScrape.parse.abundance.open_and_parse}}\pysiglinewithargsret{\sphinxcode{\sphinxupquote{pyTOPSScrape.parse.abundance.}}\sphinxbfcode{\sphinxupquote{open\_and\_parse}}}{\emph{\DUrole{n}{path}}}{}
\sphinxAtStartPar
Open and parse the contents of a chemical composition file
\begin{quote}\begin{description}
\item[{Parameters}] \leavevmode
\sphinxAtStartPar
\sphinxstyleliteralstrong{\sphinxupquote{path}} (\sphinxstyleliteralemphasis{\sphinxupquote{str}}) \textendash{} Path to open file

\item[{Returns}] \leavevmode
\sphinxAtStartPar

\sphinxAtStartPar
\sphinxstylestrong{parsed} \textendash{}

\sphinxAtStartPar
Dictionary with two indexes.
\begin{itemize}
\item {} \begin{description}
\item[{Abundance Ratio}] \leavevmode
\sphinxAtStartPar
Includes the indexes:
\begin{itemize}
\item {} 
\sphinxAtStartPar
STD (\sphinxstyleemphasis{str})

\item {} 
\sphinxAtStartPar
{[}Fe/H{]} (\sphinxstyleemphasis{float})

\item {} 
\sphinxAtStartPar
{[}alpha/Fe{]} (\sphinxstyleemphasis{float})

\item {} 
\sphinxAtStartPar
{[}C/Fe{]} (\sphinxstyleemphasis{float})

\item {} 
\sphinxAtStartPar
{[}N/Fe{]} (\sphinxstyleemphasis{float})

\item {} 
\sphinxAtStartPar
{[}O/Fe{]} (\sphinxstyleemphasis{float})

\item {} 
\sphinxAtStartPar
{[}r/Fe{]} (\sphinxstyleemphasis{float})

\item {} 
\sphinxAtStartPar
{[}s/Fe{]} (\sphinxstyleemphasis{float})

\item {} 
\sphinxAtStartPar
C/O (\sphinxstyleemphasis{float})

\item {} 
\sphinxAtStartPar
X (\sphinxstyleemphasis{float})

\item {} 
\sphinxAtStartPar
Y (\sphinxstyleemphasis{float})

\item {} 
\sphinxAtStartPar
Z (\sphinxstyleemphasis{float})

\end{itemize}

\end{description}

\item {} \begin{description}
\item[{RelativeAbundance}] \leavevmode
\sphinxAtStartPar
Includes an index for each chemical symbol given in the
file format definition provided in the module
documentation. These are all floats.

\end{description}

\end{itemize}


\item[{Return type}] \leavevmode
\sphinxAtStartPar
dict

\end{description}\end{quote}

\end{fulllineitems}

\index{open\_chm\_file() (in module pyTOPSScrape.parse.abundance)@\spxentry{open\_chm\_file()}\spxextra{in module pyTOPSScrape.parse.abundance}}

\begin{fulllineitems}
\phantomsection\label{\detokenize{pyTOPSScrape.parse:pyTOPSScrape.parse.abundance.open_chm_file}}\pysiglinewithargsret{\sphinxcode{\sphinxupquote{pyTOPSScrape.parse.abundance.}}\sphinxbfcode{\sphinxupquote{open\_chm\_file}}}{\emph{\DUrole{n}{path}}}{}
\sphinxAtStartPar
Open a chemical composition file (format defined in the module
documentation). Split the contents by line then remove all lines which
start with \#. Finally split each line by both whitespace and commas.
\begin{quote}\begin{description}
\item[{Parameters}] \leavevmode
\sphinxAtStartPar
\sphinxstyleliteralstrong{\sphinxupquote{path}} (\sphinxstyleliteralemphasis{\sphinxupquote{str}}) \textendash{} Path to file to open

\item[{Returns}] \leavevmode
\sphinxAtStartPar
\sphinxstylestrong{contents} \textendash{} List of list of strings. The outter index selects the row, the
inner index selectes the column within the row.

\item[{Return type}] \leavevmode
\sphinxAtStartPar
list

\end{description}\end{quote}

\end{fulllineitems}

\index{parse() (in module pyTOPSScrape.parse.abundance)@\spxentry{parse()}\spxextra{in module pyTOPSScrape.parse.abundance}}

\begin{fulllineitems}
\phantomsection\label{\detokenize{pyTOPSScrape.parse:pyTOPSScrape.parse.abundance.parse}}\pysiglinewithargsret{\sphinxcode{\sphinxupquote{pyTOPSScrape.parse.abundance.}}\sphinxbfcode{\sphinxupquote{parse}}}{\emph{\DUrole{n}{contents}\DUrole{p}{:} \DUrole{n}{list}}}{{ $\rightarrow$ dict}}
\sphinxAtStartPar
Parse chem file in the format described in the module documentation.

\sphinxAtStartPar
The abuundance ratios and abundances on the first row are added to a dict
under the key {[}‘AbundanceRatio’{]} and sub indexed by the comments above each
entry (Note that these are not read; rather, they are assumed to be the
same in every file). The subsequent values (on all other rows) are added to
the same dict under the key {[}‘RelativeAbundance’{]} and sub indexed by their
chemical symbols.
\begin{quote}\begin{description}
\item[{Parameters}] \leavevmode
\sphinxAtStartPar
\sphinxstyleliteralstrong{\sphinxupquote{contents}} (\sphinxstyleliteralemphasis{\sphinxupquote{list}}) \textendash{} List of list of strings. The outter index selects the row, the
inner index selected the column in the row and at each coordinate
is a string which can be cast as a float. The one exception is that
string at 0,0 is a charectar.

\item[{Returns}] \leavevmode
\sphinxAtStartPar

\sphinxAtStartPar
\sphinxstylestrong{extracted} \textendash{}

\sphinxAtStartPar
Dictionary with two indexes.
\begin{itemize}
\item {} \begin{description}
\item[{Abundance Ratio}] \leavevmode
\sphinxAtStartPar
Includes the indexes:
\begin{itemize}
\item {} 
\sphinxAtStartPar
STD (\sphinxstyleemphasis{str})

\item {} 
\sphinxAtStartPar
{[}Fe/H{]} (\sphinxstyleemphasis{float})

\item {} 
\sphinxAtStartPar
{[}alpha/Fe{]} (\sphinxstyleemphasis{float})

\item {} 
\sphinxAtStartPar
{[}C/Fe{]} (\sphinxstyleemphasis{float})

\item {} 
\sphinxAtStartPar
{[}N/Fe{]} (\sphinxstyleemphasis{float})

\item {} 
\sphinxAtStartPar
{[}O/Fe{]} (\sphinxstyleemphasis{float})

\item {} 
\sphinxAtStartPar
{[}r/Fe{]} (\sphinxstyleemphasis{float})

\item {} 
\sphinxAtStartPar
{[}s/Fe{]} (\sphinxstyleemphasis{float})

\item {} 
\sphinxAtStartPar
C/O (\sphinxstyleemphasis{float})

\item {} 
\sphinxAtStartPar
X (\sphinxstyleemphasis{float})

\item {} 
\sphinxAtStartPar
Y (\sphinxstyleemphasis{float})

\item {} 
\sphinxAtStartPar
Z (\sphinxstyleemphasis{float})

\end{itemize}

\end{description}

\item {} \begin{description}
\item[{RelativeAbundance}] \leavevmode
\sphinxAtStartPar
Includes an index for each chemical symbol given in the
file format from the module documentation. These are all
floats.

\end{description}

\end{itemize}


\item[{Return type}] \leavevmode
\sphinxAtStartPar
dict

\end{description}\end{quote}

\end{fulllineitems}

\index{parse\_abundance\_map() (in module pyTOPSScrape.parse.abundance)@\spxentry{parse\_abundance\_map()}\spxextra{in module pyTOPSScrape.parse.abundance}}

\begin{fulllineitems}
\phantomsection\label{\detokenize{pyTOPSScrape.parse:pyTOPSScrape.parse.abundance.parse_abundance_map}}\pysiglinewithargsret{\sphinxcode{\sphinxupquote{pyTOPSScrape.parse.abundance.}}\sphinxbfcode{\sphinxupquote{parse\_abundance\_map}}}{\emph{\DUrole{n}{path}\DUrole{p}{:} \DUrole{n}{str}}}{{ $\rightarrow$ numpy.ndarray}}
\sphinxAtStartPar
Parse Hydrogen, Helium, and metal mass fraction out of a csv where each row
is one composition, the first column is X, second is Y, and the third is Z.
Comments may be included in the file if the first non white space character
on the line is a hash.
\begin{quote}\begin{description}
\item[{Parameters}] \leavevmode
\sphinxAtStartPar
\sphinxstyleliteralstrong{\sphinxupquote{path}} (\sphinxstyleliteralemphasis{\sphinxupquote{str}}) \textendash{} Path to the abundance map. This should be an ascii file where each
row contains X, Y, and Z (comma delimited with no white space).
Each row will define one set of tables to be queried with the idea
being that the entire file describes the entire set of tables to
be queried.

\item[{Returns}] \leavevmode
\sphinxAtStartPar
\sphinxstylestrong{pContents} \textendash{} numpy array of all the compositions of length n where n is the
number of rows whos first non white space character was not a hash.
For a DSEP n=126. Along the second axis the first column is X, the
second is Y, and the third is Z.

\item[{Return type}] \leavevmode
\sphinxAtStartPar
np.ndarray(shape=(n,3))

\end{description}\end{quote}

\end{fulllineitems}



\section{Module contents}
\label{\detokenize{pyTOPSScrape.parse:module-pyTOPSScrape.parse}}\label{\detokenize{pyTOPSScrape.parse:module-contents}}\index{module@\spxentry{module}!pyTOPSScrape.parse@\spxentry{pyTOPSScrape.parse}}\index{pyTOPSScrape.parse@\spxentry{pyTOPSScrape.parse}!module@\spxentry{module}}

\chapter{pyTOPSScrape.scripts package}
\label{\detokenize{pyTOPSScrape.scripts:pytopsscrape-scripts-package}}\label{\detokenize{pyTOPSScrape.scripts::doc}}

\section{Module contents}
\label{\detokenize{pyTOPSScrape.scripts:module-pyTOPSScrape.scripts}}\label{\detokenize{pyTOPSScrape.scripts:module-contents}}\index{module@\spxentry{module}!pyTOPSScrape.scripts@\spxentry{pyTOPSScrape.scripts}}\index{pyTOPSScrape.scripts@\spxentry{pyTOPSScrape.scripts}!module@\spxentry{module}}

\chapter{pyTOPSScrape.misc package}
\label{\detokenize{pyTOPSScrape.misc:pytopsscrape-misc-package}}\label{\detokenize{pyTOPSScrape.misc::doc}}

\section{Subpackages}
\label{\detokenize{pyTOPSScrape.misc:subpackages}}

\subsection{pyTOPSScrape.misc.dataFiles package}
\label{\detokenize{pyTOPSScrape.misc.dataFiles:pytopsscrape-misc-datafiles-package}}\label{\detokenize{pyTOPSScrape.misc.dataFiles::doc}}

\subsubsection{Module contents}
\label{\detokenize{pyTOPSScrape.misc.dataFiles:module-pyTOPSScrape.misc.dataFiles}}\label{\detokenize{pyTOPSScrape.misc.dataFiles:module-contents}}\index{module@\spxentry{module}!pyTOPSScrape.misc.dataFiles@\spxentry{pyTOPSScrape.misc.dataFiles}}\index{pyTOPSScrape.misc.dataFiles@\spxentry{pyTOPSScrape.misc.dataFiles}!module@\spxentry{module}}

\section{Submodules}
\label{\detokenize{pyTOPSScrape.misc:submodules}}

\section{pyTOPSScrape.misc.utils module}
\label{\detokenize{pyTOPSScrape.misc:module-pyTOPSScrape.misc.utils}}\label{\detokenize{pyTOPSScrape.misc:pytopsscrape-misc-utils-module}}\index{module@\spxentry{module}!pyTOPSScrape.misc.utils@\spxentry{pyTOPSScrape.misc.utils}}\index{pyTOPSScrape.misc.utils@\spxentry{pyTOPSScrape.misc.utils}!module@\spxentry{module}}
\sphinxAtStartPar
\sphinxstylestrong{Author:} Thomas M. Boudreaux

\sphinxAtStartPar
\sphinxstylestrong{Created:} Febuary 2021

\sphinxAtStartPar
\sphinxstylestrong{Last Modified:} July 2021

\sphinxAtStartPar
Opacity utility functions


\subsection{Functions}
\label{\detokenize{pyTOPSScrape.misc:functions}}\begin{itemize}
\item {} \begin{description}
\item[{get\_target\_log\_R}] \leavevmode
\sphinxAtStartPar
Return a numpy array with the LogR values required by DSEP for high temperature opacity tables.

\end{description}

\item {} \begin{description}
\item[{get\_target\_log\_T}] \leavevmode
\sphinxAtStartPar
Return a numpy array with the LogT values required by DSEP for high temperature opacity tables.

\end{description}

\end{itemize}
\index{get\_target\_log\_R() (in module pyTOPSScrape.misc.utils)@\spxentry{get\_target\_log\_R()}\spxextra{in module pyTOPSScrape.misc.utils}}

\begin{fulllineitems}
\phantomsection\label{\detokenize{pyTOPSScrape.misc:pyTOPSScrape.misc.utils.get_target_log_R}}\pysiglinewithargsret{\sphinxcode{\sphinxupquote{pyTOPSScrape.misc.utils.}}\sphinxbfcode{\sphinxupquote{get\_target\_log\_R}}}{}{{ $\rightarrow$ numpy.ndarray}}
\sphinxAtStartPar
Get the ndarray for the LogR values that DSEP expects
\begin{quote}\begin{description}
\item[{Returns}] \leavevmode
\sphinxAtStartPar
\sphinxstylestrong{targetLogR} \textendash{} Array of LogR values expected by DSEP in opacity table

\item[{Return type}] \leavevmode
\sphinxAtStartPar
np.ndarray

\end{description}\end{quote}

\end{fulllineitems}

\index{get\_target\_log\_T() (in module pyTOPSScrape.misc.utils)@\spxentry{get\_target\_log\_T()}\spxextra{in module pyTOPSScrape.misc.utils}}

\begin{fulllineitems}
\phantomsection\label{\detokenize{pyTOPSScrape.misc:pyTOPSScrape.misc.utils.get_target_log_T}}\pysiglinewithargsret{\sphinxcode{\sphinxupquote{pyTOPSScrape.misc.utils.}}\sphinxbfcode{\sphinxupquote{get\_target\_log\_T}}}{}{{ $\rightarrow$ numpy.ndarray}}
\sphinxAtStartPar
Get the ndarray for the LogT values that DSEP expects
\begin{quote}\begin{description}
\item[{Returns}] \leavevmode
\sphinxAtStartPar
\sphinxstylestrong{targetLogT} \textendash{} Array of LogT values expected by DSEP in opacity table

\item[{Return type}] \leavevmode
\sphinxAtStartPar
np.ndarray

\end{description}\end{quote}

\end{fulllineitems}

\index{load\_non\_rect\_map() (in module pyTOPSScrape.misc.utils)@\spxentry{load\_non\_rect\_map()}\spxextra{in module pyTOPSScrape.misc.utils}}

\begin{fulllineitems}
\phantomsection\label{\detokenize{pyTOPSScrape.misc:pyTOPSScrape.misc.utils.load_non_rect_map}}\pysiglinewithargsret{\sphinxcode{\sphinxupquote{pyTOPSScrape.misc.utils.}}\sphinxbfcode{\sphinxupquote{load\_non\_rect\_map}}}{}{{ $\rightarrow$ numpy.ndarray}}
\sphinxAtStartPar
Load the upper non rectabtular map from numpy binary which DSEP requires
for the high temperature opacity files.
\begin{quote}\begin{description}
\item[{Returns}] \leavevmode
\sphinxAtStartPar
\sphinxstylestrong{upperNonRect} \textendash{} Upper non rectangular map which DSEP requires.

\item[{Return type}] \leavevmode
\sphinxAtStartPar
np.ndarray

\end{description}\end{quote}

\end{fulllineitems}



\section{Module contents}
\label{\detokenize{pyTOPSScrape.misc:module-pyTOPSScrape.misc}}\label{\detokenize{pyTOPSScrape.misc:module-contents}}\index{module@\spxentry{module}!pyTOPSScrape.misc@\spxentry{pyTOPSScrape.misc}}\index{pyTOPSScrape.misc@\spxentry{pyTOPSScrape.misc}!module@\spxentry{module}}

\renewcommand{\indexname}{Python Module Index}
\begin{sphinxtheindex}
\let\bigletter\sphinxstyleindexlettergroup
\bigletter{p}
\item\relax\sphinxstyleindexentry{pyTOPSScrape.api}\sphinxstyleindexpageref{pyTOPSScrape.api:\detokenize{module-pyTOPSScrape.api}}
\item\relax\sphinxstyleindexentry{pyTOPSScrape.api.api}\sphinxstyleindexpageref{pyTOPSScrape.api:\detokenize{module-pyTOPSScrape.api.api}}
\item\relax\sphinxstyleindexentry{pyTOPSScrape.api.convert}\sphinxstyleindexpageref{pyTOPSScrape.api:\detokenize{module-pyTOPSScrape.api.convert}}
\item\relax\sphinxstyleindexentry{pyTOPSScrape.api.utils}\sphinxstyleindexpageref{pyTOPSScrape.api:\detokenize{module-pyTOPSScrape.api.utils}}
\item\relax\sphinxstyleindexentry{pyTOPSScrape.err}\sphinxstyleindexpageref{pyTOPSScrape.err:\detokenize{module-pyTOPSScrape.err}}
\item\relax\sphinxstyleindexentry{pyTOPSScrape.err.err}\sphinxstyleindexpageref{pyTOPSScrape.err:\detokenize{module-pyTOPSScrape.err.err}}
\item\relax\sphinxstyleindexentry{pyTOPSScrape.ext}\sphinxstyleindexpageref{pyTOPSScrape.ext:\detokenize{module-pyTOPSScrape.ext}}
\item\relax\sphinxstyleindexentry{pyTOPSScrape.ext.utils}\sphinxstyleindexpageref{pyTOPSScrape.ext:\detokenize{module-pyTOPSScrape.ext.utils}}
\item\relax\sphinxstyleindexentry{pyTOPSScrape.misc}\sphinxstyleindexpageref{pyTOPSScrape.misc:\detokenize{module-pyTOPSScrape.misc}}
\item\relax\sphinxstyleindexentry{pyTOPSScrape.misc.dataFiles}\sphinxstyleindexpageref{pyTOPSScrape.misc.dataFiles:\detokenize{module-pyTOPSScrape.misc.dataFiles}}
\item\relax\sphinxstyleindexentry{pyTOPSScrape.misc.utils}\sphinxstyleindexpageref{pyTOPSScrape.misc:\detokenize{module-pyTOPSScrape.misc.utils}}
\item\relax\sphinxstyleindexentry{pyTOPSScrape.parse}\sphinxstyleindexpageref{pyTOPSScrape.parse:\detokenize{module-pyTOPSScrape.parse}}
\item\relax\sphinxstyleindexentry{pyTOPSScrape.parse.abundance}\sphinxstyleindexpageref{pyTOPSScrape.parse:\detokenize{module-pyTOPSScrape.parse.abundance}}
\item\relax\sphinxstyleindexentry{pyTOPSScrape.scripts}\sphinxstyleindexpageref{pyTOPSScrape.scripts:\detokenize{module-pyTOPSScrape.scripts}}
\end{sphinxtheindex}

\renewcommand{\indexname}{Index}
\printindex
\end{document}